%oneside para una impresi\'on simple.
%titlepage define que vamos a tener una portada.
%openany para que los cap\'{\i}tulos empiecen en cualquier p\'agina y no s\'olo las impares.
%final es para compilar a modo normal. Cambiar a draft para que se compile en modo borrador. No se pone final dado que viene por defecto.
\documentclass[a4paper, 12pt, oneside, titlepage, openany]{book}

% Para el seccionado, debe colocarse al principio este package.
\usepackage{titlesec}

% Encoding + cmap (para obtener un mapeo UTF8 adecuado)
\usepackage{cmap}
\usepackage[utf8]{inputenc}
\usepackage[T1]{fontenc} % Para poder copiar y pegar el texto desde un pdf
\usepackage{lmodern}

\usepackage[table,usenames]{xcolor}

\usepackage{amssymb} % provee \checkmark y \times
\newcommand{\cmark}{\ensuremath{\checkmark}}
\newcommand{\xmark}{\ensuremath{\times}}

\usepackage[backend=biber, style=iso-authoryear, language=spanish]{biblatex}
\DeclareUnicodeCharacter{202F}{~} % Para reemplazar el caracter 202F por un espacio
\addbibresource{biblio.bib}
\setcounter{tocdepth}{3}
\setcounter{secnumdepth}{3}
% Esto último es para la cantidad de niveles en el índice

\DefineBibliographyStrings{spanish}{
  online   = {en línea},
  urlfrom  = {Disponible en:},
  andothers = {\mkbibemph{et\addabbrvspace al\adddot}}
}

\DeclareFieldFormat{urldate}{\mkbibbrackets{consultado\space#1}}

% Para colores (ya incluido en xcolor arriba)
% \usepackage[usenames]{color} % Comentado para evitar conflicto con xcolor

% Para flotar figuras y tablas
\usepackage{framed}

% Para asegurar que las figuras y tablas queden en la secci\'on correspondiente
\usepackage[section,above,below]{placeins}

% Para microtipograf\'{\i}a
% \usepackage{microtype} % Moved below biblatex to avoid patch error

% Permite tama\~no de fuentes arbitrarios
\usepackage{anyfontsize}

% Para resaltado de texto
%\usepackage{soul}

% Itemize
\usepackage{enumitem}
\newlist{Properties}{enumerate}{2}
\setlist[Properties]{label=Propiedad \arabic*.,itemindent=*}
\renewcommand{\labelitemii}{\textasteriskcentered}
\renewcommand{\labelitemiii}{-}

\usepackage{lastpage} % Para poder contar las p\'aginas, habilita el \pageref command.
\usepackage{titleps} % Para la tabla de contenidos
\usepackage{textcase}
\usepackage{pdfpages}

% \parbox[b] es para poner texto sobre el bottom
\newpagestyle{ruled}{
	\sethead{\includegraphics[width=3.7cm]{./images/UADE_LARGE}}
			{\parbox[b]{0.43\textwidth}{\raggedright\MakeUppercase{Smartstocker: Plataforma de predicción de ventas y optimización de inventario}}} % El parámetro 0.43 puede modificarse para tener más espacio en la columan del medio, y que así el título de la tesis entre mejor.
			{
				\raisebox{1.5ex}{%
					\parbox[b]{0.26\textwidth}{\raggedleft
					\begin{tabular}{r}
						Cano, Nicolás \\
						Venegas, Gabriela
					\end{tabular}
					}
				}
			}\headrule{}
	\setfoot[][][P\'agina \thepage\ de~\pageref*{LastPage}]{}{}{P\'agina \thepage\ de~\pageref*{LastPage}}\footrule{}
	}
\pagestyle{ruled}
% Cambiar el color de la linea divisora:
% \renewcommand\makeheadrule{\color{cyan}\rule[-.3\baselineskip]{\linewidth}{0.4pt}}
% \renewcommand\makefootrule{\color{cyan}\rule[\baselineskip]{\linewidth}{0.4pt}}

% Para que la primera página de cada capítulo no sea tratada como especial
\makeatletter
\let\ps@plain\ps@ruled
\makeatother

% M\'argenes
\usepackage{vmargin}
%\setmarginsrb{leftmargin}{topmargin}{rightmargin}{bottommargin}%
%         {headheight}{headsep}{footheight}{footskip}           %
%Lo ideal ser\'{\i}a lo siguiente
%\setmarginsrb{2.7cm}{4cm}{2.3cm}{2.5cm}{2cm}{0.5cm}{2cm}{0.5cm}
%Pero estos valores se suman, por lo que en realidad ir\'{\i}a asi:
\setmarginsrb{2.7cm}{2cm}{2.3cm}{1.5cm}{0cm}{2cm}{0cm}{1cm}
% M\'argenes, pero otra opci\'on mas acotada
% \usepackage[top=4cm,bottom=2.5cm,left=2.7cm,right=2.3cm]{geometry}

% Seccionado. El par\'ametro left incrementa el margen, lo setteo en 0.
%\titlespacing*{<command>}{<left>}{<before-sep>}{<after-sep>}
% \titlespacing{\section}{0pt}{12pt}{6pt}
%\titlespacing{\subsection}{0pt}{*0}{*0}
%\titlespacing{\subsubsection}{0pt}{*0}{*0}

%Tama\~no de letra de cap\'{\i}tulo y secciones
%\newcommand{\chapfnt}{\fontsize{16}{19}}
%\newcommand{\secfnt}{\fontsize{14}{17}}
%\newcommand{\ssecfnt}{\fontsize{12}{14}}
%\titleformat{\chapter}[display]
%{\normalfont\chapfnt\bfseries}{\chaptertitlename\ \thechapter}{20pt}{\chapfnt}
%\titleformat{\section}
%{\normalfont\secfnt\bfseries}{\thesection}{1em}{}
%\titleformat{\subsection}
%{\normalfont\ssecfnt\bfseries}{\thesubsection}{1em}{}

% Se pide numeraci\'on de tablas en n\'umeros romanos
\renewcommand{\thetable}{\thechapter.\Roman{table}}

%Espaciado antes y despu\'es de una figura respecto del texto
\setlength{\textfloatsep}{5pt}

% Para notas
\usepackage{todonotes}
\setlength{\marginparwidth}{2cm} % El margen es muy estrecho. Esto es solo para solucionar un warning
\newcommand{\Nico}[1]{\todo[color=green!30,inline]{\textbf{Nico:} #1}}
\newcommand{\NicoNI}[1]{\todo[color=green!30]{\textbf{Nico:} #1}}

%Espaciado entre referencias en la bibliograf\'{\i}a
\usepackage{etoolbox}
\patchcmd{\thebibliography}
  {\settowidth}
  {\setlength{\itemsep}{6pt}\settowidth}
  {}{}
\apptocmd{\thebibliography}
  {\small}
  {}{}

\usepackage{tocloft}
\cftsetindents{table}{0em}{4em} % Ajusta 4em seg\'un sea necesario

\usepackage{pdfcomment}
\usepackage[spanish,es-nodecimaldot]{babel}
\usepackage{csquotes}
\addto\captionsspanish{
	\renewcommand{\contentsname}{\'Indice}
	\renewcommand{\listfigurename}{Lista de Figuras}
	\renewcommand{\listtablename}{Lista de Tablas}
	\renewcommand{\tablename}{TABLA}
}

% Para que el primer parrafo de un cap\'{\i}tulo tenga identado.
\usepackage{indentfirst}
\setlength{\parindent}{36pt} %Tama\~no de la identaci\'on.

% Para tablas complejas
\usepackage{multicol,multirow}
\usepackage{array,longtable}

\usepackage{amsmath,amsfonts,amssymb,amsthm}

\allowdisplaybreaks{} % Para que el align permita dividir entre p\'aginas si lo llega a necesitar.

\usepackage{cancel}

\newenvironment{example}[1][Ejemplo]{\begin{trivlist}
	\item[\hspace{\labelsep} {\itshape{} #1}]}{\end{trivlist}}

\newenvironment{examples}[1][Ejemplos]{\begin{trivlist}
	\item[\hspace{\labelsep} {\itshape{} #1}]}{\end{trivlist}}

\newenvironment{EstadoDelArte}
    {\small\begin{center}
    \bfseries{EstadoDelArte} \end{center}}

\newenvironment{Enfoque metodologico}
    {\small\begin{center}
    \bfseries{Enfoque metodologico} \end{center}}

\newenvironment{Cronograma}
    {\small\begin{center}
    \bfseries{Cronograma} \end{center}}

% Para gr\'aficos y figuras
\usepackage{tikz}
\usepackage{graphicx}
\usepackage{wrapfig}
\usepackage{blochsphere}

% Diagramas y figuras.
\usepackage{tikz-cd}
\usepackage{quiver}

% Tablas
\newcommand\titulo[3][\scriptsize]{\rotatebox[origin=c]{90}{\parbox[t]{#2}{\centering #1{#3}}}} % Estos son los t\'{\i}tulos que van al costado de las tablas que tienen todo su contenido sin dividirse.
% \newcommand\rulestitlehalf[1]{\omit\rlap{\parbox{0.3\linewidth}{\centering\textbf{#1}}}}

\usepackage{microtype} % Load microtype after biblatex
\usepackage[normalem]{ulem}

\usepackage{setspace}
\onehalfspacing

% En UADE no existe la noción de capítulo en la tesis, por más que en LaTeX sí lo tratemos como tal.
% Además nos piden que el tamaño de letra de lo que para nosotros es una sección, subsección y demás, sean del mismo tamaño.
\titleformat{\chapter}
  {\bfseries\fontsize{14pt}{14pt}\selectfont}
  {\thechapter}{1em}{}
\titlespacing*{\chapter}{0pt}{12pt}{6pt}

\titleformat{\section}
  {\bfseries\fontsize{14pt}{14pt}\selectfont}
  {\thesection}{1em}{}
\titlespacing*{\section}{0pt}{12pt}{6pt}

\titleformat{\subsection}
  {\bfseries\fontsize{14pt}{14pt}\selectfont}
  {\thesubsection}{1em}{}
\titlespacing*{\subsection}{0pt}{12pt}{6pt}

\titleformat{\subsubsection}
  {\bfseries\fontsize{14pt}{14pt}\selectfont}
  {\thesubsubsection}{1em}{}
\titlespacing*{\subsubsection}{0pt}{12pt}{6pt}

% -> Acá se empieza a modificar el tamaño de los títulos autogenerados, como los de las listas de figuras y tablas
\usepackage{tocloft}

\renewcommand{\cfttoctitlefont}{\bfseries\fontsize{14pt}{16pt}\selectfont}
\setlength{\cftbeforetoctitleskip}{12pt}
\setlength{\cftaftertoctitleskip}{6pt}

% Fuente del título (14pt, negrita)
\renewcommand{\cftloftitlefont}{\bfseries\fontsize{14pt}{16pt}\selectfont}
\renewcommand{\cftlottitlefont}{\bfseries\fontsize{14pt}{16pt}\selectfont}

% Espaciado antes y después del título
\setlength{\cftbeforeloftitleskip}{12pt}
\setlength{\cftafterloftitleskip}{6pt}
\setlength{\cftbeforelottitleskip}{12pt}
\setlength{\cftafterlottitleskip}{6pt}


\usepackage[nottoc]{tocbibind}
% IMPORTANTE: En la entrega de UADE se pide en negro, pero visualmente es mejor trabajar con color
\usepackage{hyperref}
\hypersetup{
  colorlinks=true,
  linkcolor=black,
  citecolor=black,
  urlcolor=black
}
% Para la entrega de UADE:
%\usepackage[hidelinks]{hyperref}
%\hypersetup{
%    colorlinks=false,
%    pdfborder={0 0 0},
%    linktoc=none
%}

% -> Acá terminamos

\usepackage{mathptmx}

\begin{document}
\clearpage
\thispagestyle{empty}
\includepdf[
  pages=1,
  scale=1.01, %1.17
  offset=25mm -25mm,
  pagecommand={\thispagestyle{empty}},
  noautoscale,
  trim=0 0 0 0,
  clip
]{cover/caratula.pdf}
\clearpage
	\begin{titlepage}
    \centering

    {\textbf{\fontsize{18}{20}\selectfont PROYECTO FINAL DE INGENIERÍA} \par}
    \vspace{1.5cm}

    {\textbf{\fontsize{16}{18}\selectfont SmartStocker: Plataforma web para la predicción de ventas y optimización de inventario en CABA en 2025} \par}
    \vspace{0.5cm}

    {\textbf{\fontsize{14}{16}\selectfont Cano, Nicolás Martín -- LU 1147246} \par}
    {\fontsize{14}{16}\selectfont Ingeniería en Informática \par}
    \vspace{1cm}

    {\textbf{\fontsize{14}{16}\selectfont Venegas Ramírez, Gabriela Rossana -- LU 1140594} \par}
    {\fontsize{14}{16}\selectfont Ingeniería en Informática \par}
    \vspace{1.5cm}

    {\fontsize{14}{16}\selectfont Tutor: \par}
    {\textbf{\fontsize{14}{16}\selectfont Monzón, Nicolás Alberto, 
        \\ (UADE) Universidad Argentina de la Empresa, Lima 757, Cdad. Autónoma de Buenos Aires, Argentina.
        \\ (UdelaR) Universidad de la República, Av. 18 de Julio 1968, 11200 Montevideo, Departamento de Montevideo, Uruguay.
    } \par}
    \vspace{3cm}

	%{\textbf{\fontsize{14}{16}\selectfont \the\year} \par}
    % {\textbf{\fontsize{14}{16}\selectfont 2025} \par} % se puede hardcodear
	 {\textbf{\fontsize{14}{16}\selectfont \today} \par} % fecha completa si es la entrega final
    \vspace{2cm}

    \includegraphics[width=0.30\textwidth]{./images/UADE}\par \vspace{1cm}
    {\textbf{\fontsize{14}{16}\selectfont UNIVERSIDAD ARGENTINA DE LA EMPRESA} \par}
    {\fontsize{14}{16}\selectfont FACULTAD DE INGENIERÍA Y CIENCIAS EXACTAS \par}
\end{titlepage}
	%\chapter*{Agradecimientos}

Completar. \sout{Ejemplo de texto tachado}
	\newpage
	%\chapter*{Resumen}

%% Si se quiere enviar este resumen en una p\'agina a alguien, descomentar esto.

%\begin{center}
%\textbf{\large Universidad Argentina De la Empresa}

%\textbf{Facultad de Ingenier\'{\i}a y Ciencias Exactas}

%\textbf{Departamento de Tecnolog\'{\i}a Inform\'atica}

%\end{center}

%\begin{center}
%\large Ingenier\'{\i}a en Inform\'atica
%\end{center}

%\begin{center}
%\textbf{\large Título de la PFI}
%\end{center}

%\begin{center}
%Apellido, Nombres
%Apellido, Nombres
%\end{center}

%\noindent 
Acá va el resumen. \NicoNI{Este es un comentario de margen. Tener en cuenta que el breve margen que tenemos, anula un poco su utilidad. Son libres de usarlo de todas formas}

Con el siguiente comando se imprime el interlineado en el documento pdf. \\
Interlineado simple: 12pt.\\
Interlineado esperado: 18pt.\\
Interlineado exacto de este documento: \the\baselineskip.
	%\newpage
	%\chapter*{Abstract}

This Final Engineering Project proposes the development of a predictive web platform for inventory management in gastronomic establishments in the Autonomous City of Buenos Aires. The main objective is to optimize inventory stock in gastronomic SMEs by 2025 by predicting sales based on a machine learning model, reducing economic losses due to overstocking or shortages, and improving operational efficiency.

The proposal consists of a web platform aimed at gastronomic SMEs, which integrates sales data from platforms like PedidosYa and uses machine learning models to estimate future demand. Unlike existing commercial solutions in the Argentine market, the system allows the incorporation of external variables (weather, holidays) and user feedback to adjust its predictions.

	%\newpage
	\tableofcontents
	\chapter{Introducción}

El sector gastronómico ha sido identificado como un área estratégica dentro de la economía de la Ciudad Autónoma de Buenos Aires, destacándose por su capacidad de transformar barrios a través de la generación masiva de empleos y una variada oferta culinaria, demostrando que esta actividad no solo es fundamental para el turismo, sino también para la vida cotidiana de los residentes del país \parencite{srsur2024gastronomico}. En línea con esta relevancia, la industria gastronómica en Argentina es una de las más dinámicas y resilientes a la situación económica. A pesar de que indicadores de la Cámara Argentina de Comercio y Servicios (CAC) \parencite{cac2024consumo} determinaron que el consumo en el país cerró el 2024 con una caída acumulada de -7,4\%, en el presente el sector gastronómico experimenta un incremento constante que apunta hacia la recuperación. Pero aun cuando tanto empresarios como operadores culinarios han utilizado estrategias financieras y alternativas creativas para tratar de superar las barreras que impone la inestabilidad económica del país, sigue siendo necesario encontrar mecanismos que eleven la eficiencia operativa y así conseguir un punto de equilibrio entre la rentabilidad y la sostenibilidad.

Por lo tanto, ante el panorama presentado los distintos establecimientos gastronómicos buscan la optimización de sus operaciones, apalancándose en soluciones tecnológicas que permitan gestionar aspectos tales como la venta y facturación de sus productos, la gestión de inventario, o análisis y reportes del funcionamiento del negocio, es decir, la tecnología pasa a tener un rol preponderante, ayudando a la toma de decisiones informadas que permitan maximizar la ganancia y reducir los costos \parencite{agrawal2023predictive}.

En el presente contexto, los avances relacionados a la tecnología predictiva, especialmente aquellos que comprenden modelos de Machine Learning, son capaces de brindar una solución innovadora. Como lo demuestra \parencite{schmidt2022mlsales}, los restaurantes tienen la capacidad de adaptarse a estas invenciones para aprovechar dichos modelos y así predecir la demanda.

A pesar de la implementación exponencial de la informática en múltiples áreas de la industria argentina, la predicción de ventas en el rubro gastronómico sigue sin ser explotada, especialmente por fuera de sistemas propietarios.

Brindarle esta posibilidad a un negocio significa disminuir enormemente la incertidumbre alrededor del cálculo de stock requerido. Esta es una tarea crítica, puesto que comprar por encima de lo necesario, en el caso de los ítems orgánicos con una fecha de vencimiento acotada, se traduce directamente en pérdidas económicas por desperdicio. Y de forma similar, comprar por debajo continúa generando un impacto financiero negativo debido a las ventas que no pueden concretarse, pero también impacta a la reputación del negocio y la fidelización del cliente \parencite{agrawal2023predictive}.

Con fundamento en consultas efectuadas a dueños y encargados de negocios, hoy en día esta es una tarea solamente realizada a partir de la experiencia personal y basada en los datos de ventas de la última semana, no considerando aspectos externos tales como la estacionalidad o el clima, cuando se encuentra demostrado que estos son determinantes a la hora de realizar predicciones en esta industria \parencite{tanizaki2019forecasting}.

Actualmente, se encuentran dos principales líderes de mercado en Argentina, siendo estos Maxirest y Restô, los cuales a pesar de que brindan soluciones integrales para la gestión del negocio, no atacan esta problemática en particular. Se enfocan únicamente en un análisis descriptivo de las ventas o el inventario, es decir, no abordan de forma específica la problemática de predicción de almacenamiento de stock y por ende tampoco de venta.  

Maxirest es un software integral de gestión gastronómica el cual incluye funcionalidades como reportes en tiempo real, control de inventario y caja, además de ofrecer integración con plataformas de delivery como PedidosYa o Rappi, y medios de pago como MercadoPago o Payway.  

Del mismo modo, Restô es un sistema escalable de gestión gastronómica que facilita la gestión de inventario, compras, cuentas corrientes y emisión de documentos electrónicos. Igualmente, incluye múltiples módulos administrativos y contables, junto con la integración de distintos modos de pago y delivery. 

Sin embargo, ninguno de ellos comprende capacidades de predicción de demanda mediante Machine Learning, ni estimación automática de stock basado en dichas predicciones, así como tampoco permiten el ajuste dinámico del modelo a través de la retroalimentación del usuario. En consecuencia, limitan la capacidad de los negocios para la toma de decisiones basadas en estimaciones futuras, así como tampoco contemplan una personalización avanzada según el contexto de cada local.  

Ante esta situación, se propone la creación de una plataforma web que permita a un negocio, mediante el uso de un modelo de Machine Learning, obtener predicciones de ventas, y del inventario requerido para lograr satisfacerla, es decir, ir un nivel más allá de los líderes del mercado, ofreciendo un análisis predictivo mediante el uso de tecnologías emergentes basadas en Machine Learning \parencite{posch2022bayesian}.

A fin de entrenar el modelo de Machine Learning, se utilizará la información de ventas del negocio, realizando integraciones con los sistemas de POS a fin de obtenerla en tiempo real, y a su vez, se permitirá que el usuario brinde feedback sobre las predicciones, siendo este usado para continuar ajustando el modelo a las particularidades del negocio \parencite{soto2024futuro}.

La solución está orientada a PYMES gastronómicas ubicadas en CABA, que busquen reducir la incertidumbre a la hora de tomar decisiones respecto a la gestión de su stock, utilizando para esto la información de ventas.

\section{Objetivo General}

Permitir la optimización de los niveles de stock en el rubro gastronómico de Argentina en el año 2025, mediante predicción de ventas utilizando un modelo de Machine Learning, en base a la información de ventas, a fin de reducir costos y maximizar ingresos.

\noindent\textbf{Objetivos Específicos:}

\begin{itemize}
    \item Entrevistar a dueños y/o encargados de locales gastronómicos a fin de comprender el proceso de gestión de inventario y qué factores afectan a las ventas.
    \item Desarrollar integraciones con sistemas de plataformas de delivery, tales como PedidosYa, a fin de obtener y unificar las ventas en tiempo real.
    \item Desarrollar un modelo de Machine Learning que permita pronosticar ventas y extrapolar el inventario mínimo necesario para satisfacer la demanda.
    \item Usar fuentes externas de datos, tales como APIs meteorológicas o de feriados, a fin de enriquecer los datos a usar para la predicción.
    \item Permitir que el usuario defina alertas que notifique cuando un ítem del inventario queda por debajo de un umbral.
    \item Permitir que el usuario brinde feedback sobre estos pronósticos, a fin de permitir que el modelo se ajuste.
    \item Desarrollar una plataforma web donde el usuario pueda cargar los productos que vende, indicando de qué elementos del inventario se encuentran compuestos.
\end{itemize}

\section{Alcance}

El proyecto tiene como objetivo el desarrollo de una herramienta integral para la predicción de ventas y gestión de inventario en restaurantes del área de CABA, mediante la integración con sistemas POS, un modelo de Machine Learning y una interfaz web orientada al usuario gastronómico. 

El sistema contempla: 

\begin{itemize}
    \item Integración con sistemas de POS para la recopilación unificada de datos de ventas. 
    \item Desarrollo y entrenamiento de un modelo predictivo de ventas utilizando Machine Learning. 
    \item Incorporación de variables externas y feedback de usuarios en el entrenamiento del modelo. 
    \item Página web para la visualización de ventas, predicciones e inventario. 
    \item Alertas sobre los niveles de inventario. 
\end{itemize}

En el primer release se incluyen los siguientes módulos: 

\begin{itemize}
    \item Módulo de Integración POS (Módulo 1): Desarrolla interfaces para unificar la información proveniente de los sistemas de POS más relevantes en CABA. 
    \item Módulo de Predicción de Ventas (Módulo 2): Entrena un modelo de Machine Learning con datos históricos de ventas y factores externos (clima, feriados, día, etc), considerando el feedback brindado por los usuarios a fin de ajustar aún más el modelo. 
    \item Módulo Web de Visualización (Módulo 3): Permite a los usuarios visualizar ventas históricas, predicciones, niveles de inventario requeridos y brindar feedback sobre las predicciones. 
    \item Módulo de Configuración de Productos (Módulo 4): Permite definir productos, asociar sus componentes de inventario y configurar los umbrales a utilizar por las alertas.
\end{itemize}

Sin embargo, es importante acotar que existen varios aspectos que no son incluidos dentro del alcance del presente proyecto, en vista de la modalidad y características del mismo. No comprende la integración con todos los sistemas POS disponibles, ya que se concentra únicamente en los utilizados dentro del sector gastronómico de CABA. Estos no sufren modificaciones o personalizaciones que alteren su funcionamiento.

Se limita al desarrollo de una plataforma web, por lo tanto, no incluye el diseño de una aplicación móvil o de otro tipo de software adicional. Asimismo, el modelo de Machine Learning se aplica solamente a productos gastronómicos. Por lo tanto, quedan excluidos otros tipos de negocios o productos que, aunque tengan características semejantes, no forman parte de este sector.

	\chapter{Antecedentes}

En esta sección del documento, se presentan definiciones fundamentales para comprender el dominio del problema de investigación dentro, así como un análisis detallado de otras investigaciones realizadas sobre el tema elegido.

\section{Marco teórico}

La presente sección proporciona un fundamento teórico esencial y necesario para
comprender los conceptos que sustentan el proyecto final de ingeniería. Se presentan
definiciones con el objetivo de que, mediante su lectura, se pueda entender de manera
conceptual todos los aspectos que forman parte de la problemática planteada y de los desarrollos
posteriores. Este apartado aborda términos y conceptos generales que no requieren un
conocimiento técnico profundo, permitiendo a los lectores familiarizarse con los conceptos
básicos que enmarcan el contexto del estudio y la problemática.

\subsection{Inventario en el rubro gastronómico}

Según \parencite{chopra2019supply}, se define al inventario como el conjunto de bienes o productos que una empresa mantiene en existencia con el propósito de satisfacer la demanda futura. Puede existir en forma de materias primas, productos en proceso o productos terminados, y se mantiene debido a razones como economías de escala, incertidumbre en la demanda o suministro, y variabilidad estacional.

El concepto de \emph{supply chain} se refiere al conjunto de procesos, actores y recursos involucrados en el flujo de productos, información y servicios desde los proveedores hasta el consumidor final. \parencite{chopra2019supply} definen la cadena de suministro como \guillemotleft \emph{todos los niveles involucrados, directa o indirectamente, en satisfacer la demanda del cliente}\guillemotright.

Focalizándose en el rubro gastronómico, esto implica gestionar la compra, almacenamiento y transformación de la materia prima requerida para los distintos productos a comercializar, de forma tal que estén disponibles en el momento que son requeridos, y con los niveles de calidad esperados por el cliente.

 A su vez, existen ciertas particularidades en las materias primas que conciernen a este rubro:

 \begin{itemize}
    \item La demanda es variable, ya que factores externos, tales como el clima o feriados, provocan cambios en el patrón de consumo de los clientes.

    \item Muchas de las materias primas son perecederas, con una vida útil muy corta, lo que implica que las compras a proveedores ocurren muy frecuentemente.

    \item A su vez, los pagos a proveedores tienden a ocurrir al menos una semana luego de la entrega de mercadería, lo que implica que cualquier desvío, ya sea en cantidad o tipo de materia prima requerida se traslada directamente como una pérdida económica, debido a la naturaleza perecedera de la misma.
\end{itemize}

Este proceso de reposición de inventario, siendo uno de los que ocurre con mayor frecuencia, es a su vez uno de los de mayor costo asociado, significando para uno de los entrevistados, en mayo de 2025, alrededor de 6 millones de pesos argentinos por semana en cada uno de sus locales.

A su vez, el estimar cuanto se deberia comprar es una tarea que hoy en día, debido a los costos asociados, queda en manos del personal con mayor conocimiento del negocio. Inclusive, muchas veces recae directamente en el dueño, quien debe considerar no solo las ventas que ya ocurrieron, sino también cómo los factores externos, como los feriados, el clima y el momento del mes, podrían influir en las ventas de la próxima semana.

\subsection{Aprendizaje supervisado}

El aprendizaje supervisado es una de las principales ramas de Machine Learning y se refiere al proceso mediante el cual un algoritmo aprende a realizar predicciones a partir de un conjunto de datos (dataset) etiquetado. Según \parencite{russell2022ai}, \guillemotleft{}\emph{en el aprendizaje supervisado, el agente observa pares de entrada-salida y aprende una función que mapea desde la entrada hacia la salida. Por ejemplo, las entradas podrían ser imágenes de una cámara, cada una acompañada de una salida que indica \textquotedblleft autobús\textquotedblright{} o \textquotedblleft peatón\textquotedblright{}. Esta salida se llama etiqueta. El agente aprende una función que, dada una nueva imagen, predice la etiqueta apropiada.}\guillemotright{}.

Formalmente, el aprendizaje supervisado parte de un conjunto de entrenamiento conformado por $n$ ejemplos etiquetados:

\[(x_1, y_1), (x_2, y_2), \ldots, (x_n, y_n)\]

\noindent donde cada $x_n$ es una entrada (también conocida como vector de características) y cada $y_n$ es la salida o etiqueta correspondiente. Estos ejemplos son generados por una función desconocida $f(x)$, y el objetivo del algoritmo es encontrar una función aproximada $h(x)$, denominada \emph{hipótesis}, que generalice correctamente sobre datos no vistos.

Los problemas que son abarcados por este tipo de aprendizaje son

 \begin{itemize}
    \item Clasificación: cuando las salidas $y_n$ son categorías discretas, tales como un tipo de producto o la decisión de si realizar o no una compra.

    \item Regresión: cuando las salidas son valores numéricos continuos (por ejemplo, predicción de precios, estimación de demanda o ventas futuras).
\end{itemize}

El problema de predicción de ventas abarcado en esta tesis entra en la categoría de problemas de regresión, ya que la variable a predecir es numérica continua.

\subsection{Métricas}

A la hora de implementar algún algoritmo predictivo, es fundamental tener la capacidad de medir cuán precisas son las predicciones devueltas. Es aquí, donde las métricas toman un rol preponderante, ya que permiten evaluar de distintas formas la performance de un algoritmo.

Para el caso específico de regresión, se utilizan métricas que calculan el grado de error entre los valores reales y los valores predichos por el modelo

Basándonos en \parencite{hyndman2018forecasting, james2013isl} podemos pasar a definir las tres métricas principales para modelos de regresión.

\begin{itemize}
    \item \textbf{MAE (Mean Absolute Error)}: Mide el promedio de las diferencias absolutas entre los valores predichos por el modelo y los valores reales observados, sin considerar la dirección del error. La unidad utilizada es la misma de lo que se está prediciendo, lo que facilita su interpretación, y es menos sensible a valores atípicos en comparación con otras métricas como MSE y RMSE. Se define como:

    \[
        \mbox{MAE} = \frac{1}{n} \sum_{t=1}^{n} \left| y_t - \hat{y}_t \right|
    \]

    \item \textbf{MSE (Mean Squared Error)}: Cuantifica la diferencia promedio entre los valores reales observados y las predicciones de un modelo. Se calcula como el promedio del cuadrado de los errores, es decir, la diferencia entre los valores reales y los valores predichos. Es muy sensible a valores atípicos, y su unidad es el cuadrado de la unidad de error original, lo que hace que no sea de fácil interpretación.
    
    \[
        \mbox{MSE} = \frac{1}{n} \sum_{t=1}^{n} \left( y_t - \hat{y}_t \right)^2 
    \]

    \item \textbf{RMSE – Root Mean Squared Error:} Similar al MSE, penaliza de igual forma a los valores atípicos, pero al aplicar una raíz cuadrada al error permite que este se mantenga en la misma escala que las unidades originales, solucionando el problema de interpretación del MSE.

    \[
        \mbox{RMSE} = \sqrt{ \frac{1}{n} \sum_{i=1}^{n} (y_i - \hat{y}_i)^2 }
    \]
\end{itemize}

\subsection{CatBoost}

Antes de definir el algoritmo CatBoost, es necesario definir dos conceptos claves para comprenderlo, siendo estos el concepto de función de pérdida, y de Gradient Boosting.

Una \emph{función de pérdida} es una función matemática que cuantifica el error entre la predicción del modelo ($\hat{y}$) y el valor real observado ($y$). Su objetivo es proporcionar una medida numérica del desempeño del modelo para cada ejemplo individual o conjunto de datos, la cual puede luego utilizarse para optimizar los parámetros del modelo mediante técnicas como el descenso del gradiente.

Formalmente, dada una función de predicción $f(x)$ y un valor objetivo $y$, una función de pérdida $L(y, f(x))$ devuelve un valor escalar que representa la penalización por la discrepancia entre ambos.

Definido el concepto de función de pérdida, podemos pasar a explicar el concepto de Gradient Boosting, siendo este una técnica de ensamble supervisado que construye un modelo predictivo fuerte mediante la combinación secuencial de múltiples modelos débiles, generalmente árboles de decisión, con el objetivo de corregir los errores residuales del conjunto anterior. A cada paso, el nuevo modelo es entrenado para minimizar el error del conjunto previo, utilizando métodos de optimización basados en el gradiente de una función de pérdida.

En este caso, la función utilizada por CatBoost es RMSE, cuya fórmula matemática ha sido definida previamente

Dada la definición de Gradient Boosting, podemos pasar a explicar CatBoost (Categorical Boosting), definido en \parencite{dorogush2018catboost}, como uno de los algoritmos implementados con esta técnica. Está diseñado específicamente para manejar datasets que contienen una combinación de variables numéricas y categóricas, destacándose por su capacidad para procesar variables categóricas sin necesidad de codificación previa (como one-hot o label encoding), y de reducir el overfitting que ocurre durante el entrenamiento mediante una técnica llamada Ordered Boosting. 

A su vez, es uno de los algoritmos de Machine Learning más performantes a la hora de trabajar con datasets de las características a utilizar en esta tesis.

\section{Estado del Arte}

Según \parencite{creswell2014}, el estado del arte constituye una instancia fundamental dentro del proceso de investigación debido a que permite conocer las soluciones vigentes a la problemática abordada, identificar los aportes realizados por otros actores y a su vez delimitar las distintas oportunidades de innovación para el proyecto en curso.

En este sentido, el presente apartado desarrolla un análisis del estado del arte en relación con el diseño de soluciones tecnológicas orientadas a la gestión inteligente del stock en el sector gastronómico, haciendo énfasis en aplicaciones que incorporan técnicas de \emph{Machine Learning} para la predicción de la demanda.

Ahora bien, con el objetivo de identificar los aportes diferenciales de la solución propuesta, se abordan tres tópicos esenciales: primeramente aplicaciones comerciales actuales, en segundo lugar proyectos académicos o experimentales relacionados, y por último una lectura estratégica basada en la teoría del Océano Azul para identificar oportunidades de innovación y creación de valor en espacios de mercado poco explorados.

\subsection{Soluciones tecnológicas con enfoque predictivo en el sector gastronómico internacional}\label{sec:estado-internacional}

A nivel mundial, la búsqueda por una gestión eficiente de los establecimientos gastronómicos ha impulsado el desarrollo de herramientas que integran inteligencia artificial, especialmente técnicas de Machine Learning. Estas soluciones han surgido en distintos países con el objetivo de anticipar la demanda y optimizar la administración de insumos. A continuación, se presentan dos de las plataformas más relevantes, con el fin de profundizar en el uso de la inteligencia artificial en el sector gastronómico a nivel global.

\begin{itemize}
    \item \textbf{5-Out}. Plataforma estadounidense que utiliza \emph{Machine Learning} para predecir ventas con integración de fuentes internas (TPV, programación de personal, reservas) y externas (clima, eventos), habilitando recomendaciones de compra para equilibrar inventario y reducir desperdicio. Es integrable con sistemas como Toast, Square y 7Shifts; además, anunció una alianza con Craftable para ampliar funciones administrativas y financieras \parencite{fiveout2025partnership}.

        Gracias al análisis predictivo, esta herramienta puede indicarle de forma detallada al sitio gastronómico la cantidad de productos que deberá comprar para mantener un equilibrio en la gestión del inventario. Esto ayuda a reducir significativamente el desperdicio de alimentos y aumenta en consecuencia la rentabilidad. 

        Además, esta aplicación es compatible e integrable a sistemas actuales como Toast, Square y 7Shifts, y consideran unirse a Craftable para sumar funciones administrativas y financieras a sus capacidades predictivas como alternativa innovadora en la gestión de restaurantes. 

    \item \textbf{Restoke}. Es una startup australiana que automatiza procesos administrativos en restaurantes mediante IA: control de inventario y compras, programación y proyección de ventas, e integración con POS y software contable. En 2024 fue noticia por su crecimiento y una ronda de financiación de USD 5,1 millones, con casos reportados de fuertes reducciones de costos operativos \parencite{santoreneos2024restoke}.
\end{itemize}

\subsection{Soluciones comerciales actuales en el sector gastronómico argentino}\label{sec:estado-arg}

Actualmente, en Argentina, el mercado del software especializado en el sector gastronómico se encuentra en constante crecimiento, que si bien todavía no ha explotado el mundo del aprendizaje automatizado, dispone de herramientas que facilitan la administración y operación diaria de distintos establecimientos como restaurantes, sitios de comida rápida, cafés y lugares afines. Es por esto que se resalta algunas de las plataformas más utilizadas dentro del presente rubro, Maxirest y Restô , ambas con un enfoque integral en la gestión del negocio.

\begin{itemize}
    \item \textbf{Maxirest} \parencite{latam2024maxirest}. Se trata de una herramienta reconocida en el mercado por su robustez y funcionalidad. Su enfoque se centra exclusivamente en el análisis descriptivo de datos históricos y es una de las soluciones más consolidadas del país en el ámbito gastronómico. La plataforma permite automatizar diversas tareas operativas, tales como la gestión de cocina, la toma de pedidos, la administración de reservas e inventario, y la generación automática de reportes. Además, incluye un módulo específico para delivery y takeaway, lo que resulta especialmente relevante considerando que el 47,6\% de las ventas del sector se realizan a través de estas dos modalidades.
    
    \item \textbf{Restô} \parencite{iprofesional2014resto}. Es un sistema de gestión gastronómica que forma parte del ecosistema del software Tango, el cual fue desarrollado por la empresa argentina Axoft. El propósito de este es brindar una solución a la necesidad de una herramienta tecnológica que simplifique los procesos del día a día de dicho sector. En línea con dicho objetivo, este brinda una administración integral en múltiples establecimientos gastronómicos con diversas funciones tales como la gestión de mesas y reservas, el control de stock de insumos, la facturación electrónica, el seguimiento de ventas, y muchas otras más. De igual manera, gracias a su enfoque modular y adaptable, es utilizado tanto por pequeños comercios como por grandes cadenas gastronómicas, lo que le ha permitido posicionarse como una de las soluciones locales más completas del mercado para la administración de negocios del rubro alimenticio.
\end{itemize}

Ahora bien, con el objetivo de sintetizar las principales características de las soluciones comerciales vigentes en Argentina y contrastarlas con la propuesta de SmartStocker, se elaboró la siguiente tabla comparativa. En ella se ponen en relación funcionalidades clave vinculadas a la gestión del inventario, los reportes en tiempo real y, especialmente, la incorporación de modelos de Machine Learning.  
\begin{table}[htbp]
  \centering
  \begin{tabular}{lccc}
    \hline
    \textbf{Funcionalidad} & \textbf{Maxirest} & \textbf{Restô} & \textbf{SmartStocker} \\
    \hline
    Reportes en tiempo real        & \cmark & \cmark & \cmark \\
    Control de inventario          & \cmark & \cmark & \cmark \\
    Integración con plataformas    & \cmark & \cmark & \cmark \\
    Facturación/gestión admin.     & \cmark & \cmark & \xmark \\
    Predicción de demanda (ML)     & \xmark & \xmark & \cmark \\
    Estimación dinámica de stock   & \xmark & \xmark & \cmark \\
    Variables externas en modelos  & \xmark & \xmark & \cmark \\
    Feedback de usuarios (ajuste)  & \xmark & \xmark & \cmark \\
    Tablero de métricas predictivas& \xmark & \xmark & \cmark \\
    Alertas preventivas de stock   & \xmark & \xmark & \cmark \\
    Enfoque exclusivo en gastronomía & \cmark & \cmark & \cmark \\
    \hline
  \end{tabular}
    \caption{Comparación de funcionalidades entre la competencia y SmartStocker.}
    \label{tab:comparacion-funcionalidades}
\end{table}
Tal como se observa, las plataformas locales se limitan a la automatización y análisis descriptivo, mientras que SmartStocker incorpora un enfoque predictivo con modelos de Machine Learning, evidenciando así el diferencial innovador del proyecto.

\subsection{Propuestas académicas y experimentales}\label{sec:academico}

Aunque el modelo predictivo se ha desarrollado en investigaciones de múltiples áreas, existen escasos trabajos académicos enfocados específicamente en el rubro gastronómico. Uno de ellos es \parencite{hari2024predictiwaste}: \emph{PredictiWaste: An ML-Powered Framework for Sustainable Food Inventory Optimization in Restaurants (marco de trabajo basado en aprendizaje automático para la optimización sostenible del inventario de alimentos en restaurantes)}, se introduce un modelo de análisis predictivo para reducir el desperdicio de alimentos en el inventario y la planificación de menús de restaurantes. 

Por consiguiente, realiza el pronóstico del desperdicio basándose en factores como el tipo y cantidad de alimento, número de comensales, tipo de evento, condiciones de almacenamiento, historial de compras, métodos de preparación, estacionalidad, ubicación y precio. El sistema clasifica el desecho como mínimo, moderado o alto y sugiere medidas de control de inventario, ayudando en consecuencia a las empresas a optimizar sus recursos y a promover la sostenibilidad. 

Igualmente, una de las propuestas más representativas se encuentra en \parencite{schmidt2022mlsales}, con el pronóstico de ventas de restaurantes basado en aprendizaje automático. Este estudio afirma que para la correcta gestión del personal en los restaurantes, es necesaria la previsión precisa de las ventas, y propone un caso práctico sobre diversos modelos de aprendizaje automático utilizando data real de las ventas de un negocio, tipo restaurante de tamaño medio, y la inclusión de modelos de redes neuronales recurrentes de tendencia para la comparación de su rendimiento a través de diversos métodos.

De la misma manera, otros estudios como el \parencite{tanizaki2019forecasting} sobre la previsión de la demanda en restaurantes mediante aprendizaje automático y análisis estadístico, construyen un modelo específico para cada tienda física que combina funcionalmente datos de diversos factores externos tales como ubicación, clima y eventos, mediante aprendizaje automático, para luego analizar el resultado de verificación con datos de otros negocios similares.

\subsection{Océano Azul}

La Estrategia del Océano Azul es una teoría innovadora conceptualizada en \parencite{kim2015estrategia} que redefine la competencia empresarial a través de la innovación, como creadora de oportunidades y crecimiento rentable en mercados inexplorados.

Según sus autores, los océanos rojos representan a todas las industrias que hoy operan en el mercado, donde sus características están altamente definidas y son aceptadas, lo que conlleva a la lucha entre empresas rivales por obtener el posicionamiento ventajoso de llevarse una mayor participación en la demanda existente. Es decir, es un nivel de subsistencia toda vez que al saturarse el mercado se reducen las perspectivas de rentabilidad y sostenibilidad, lo que ocasiona que los productos pierdan su rasgo de distinción y por ende se conviertan en genéricos.

Por el contrario, los océanos azules representan nichos de mercado no explorados donde es factible la creación de demanda con un crecimiento altamente rentable. En el contexto de los océanos azules, la competencia carece de relevancia, dado que los parámetros operativos aún no han sido definidos.

Los teóricos con la intención de cuantificar el impacto de la creación de los océanos azules sobre el crecimiento de una compañía, analizaron el lanzamiento de 108 negocios nuevos, lo que arrojó como resultado que el 86\% eran extensiones de líneas existentes, es decir, pertenecientes al océano rojo y el 14\% restante de los lanzamientos pertenecientes a océanos azules.

Esto subraya la diferencia entre la creación de nuevos mercados sin competencia que rompen el equilibrio tradicional entre valor y costos, generando crecimiento empresarial con ganancias más rentables y a su vez con la competencia tradicional directa en mercados ya existentes que se encuentran saturados por parte de empresas que luchan por el control de la demanda.

En la estrategia del océano azul, se definen como varias herramientas fundamentales a saber, la matriz de \emph{Eliminar-Reducir-Aumentar-Crear} como recurso central, que \guillemotleft obliga a las compañías no solo a perseguir simultáneamente la diferenciación y el bajo costo, sino también a romper con el compromiso de valor-costo. No se trata de crear una nueva curva de valor por crearla, sino de identificar las variables correctas que transformarán la dinámica de la industria.\guillemotright \parencite{kim2015estrategia}.

Este enfoque analítico desafía la lógica de la competencia tradicional de la industria brindando apertura a nuevos paradigmas de análisis económico, ya que impulsa a las empresas a pensar de manera diferente sobre cómo crear valor para el cliente. 

Ahora bien, la Matriz \emph{Eliminar-Reducir-Aumentar-Crear}, es la representación tabular del Marco de las Cuatro Acciones, y tiene como finalidad ayudar en la formación de las ideas para que la toma de decisiones tenga como objetivo la diferenciación y el bajo costo de manera integrada.

Asimismo, otra herramienta fundamental es la \emph{Curva de Valor}, la cual permite una visualización gráfica del perfil estratégico de una empresa en relación con los factores clave de la industria. Esta curva permite identificar si una propuesta se encuentra alejada o no, de lo convencional, permitiendo así la detección temprana de oportunidades de innovación que generen una ventaja competitiva sostenible dentro del mercado.


\subsection{Matriz ERAC}\label{sec:erac}

\begin{figure}[htbp]
    \centering
    \includegraphics[width=0.7\textwidth]{images/matrizEric.jpeg}
    \caption{Matriz ERAC de SmartStocker}
    \label{fig:eric}
\end{figure}

Las variables identificadas en la matriz ERAC (Figura~\ref{fig:eric}) se utilizan como insumo para construir la Curva de Valor (Figura~\ref{fig:curva}), con el objetivo de visualizar el posicionamiento de \emph{SmartStocker} en relación con sus competidores en el mercado.

\FloatBarrier

\clearpage
\subsection{Curva de Valor}\label{sec:curva-valor}

\begin{figure}[htbp]
    \centering
    \includegraphics[width=0.7\textwidth]{images/curvaValor.jpeg}
    \caption{Curva de valor.}
    {\textit{Fuente: Elaboración propia.}}
    \label{fig:curva}
\end{figure}

\FloatBarrier

\subsection{Conclusión}\label{sec:sintesis-estado}

Una vez realizado el análisis del estado del arte y de las herramientas que utiliza, se evidencia el diferencial del presente proyecto ya que no existen en la actualidad soluciones tecnológicas predictivas en el mercado nacional que conduzcan a una precisa gestión del inventario en el rubro gastronómico en base a la información de ventas, toda vez que las otras plataformas en el mercado se limitan a funciones descriptivas, basadas en estadísticas generales de consumo.

Asimismo, la presente tesis desarrolla un enfoque novedoso ya que utiliza el modelo de Machine Learning para predecir las cantidades de producto necesarias para el funcionamiento óptimo del usuario, ajustar los niveles de existencias en tiempo real y mantener los niveles de stock necesarios para que el local sea altamente productivo y su rentabilidad sea sostenible.

Mantener el inventario en un punto de equilibrio óptimo impacta directamente en la eficiencia operativa del negocio ya que implica la reducción de costos y la posibilidad de minimizar el capital inactivo en productos consumibles. Como consecuencia, genera grandes beneficios tales como el control de los desperdicios que puede traducirse a corto plazo en una reducción de gastos. Permite la rotación correcta de los insumos, evita la sobrecompra y la posible merma por caducidad ya que los alimentos tienen un tiempo de vida útil limitado, y luego de transcurrido este, no queda otra que descartarlos por control sanitario. Consecuencialmente, también se evita el gasto colateral que se genera en tiempo horas-hombre en personal utilizado para eliminar desechos, asear y reacondicionar el lugar.

Además, permite determinar el stock de seguridad, lo que es igual a la cantidad mínima de insumos necesarios para la producción base del negocio, facilitando así la gestión de mantenimiento de los volúmenes necesarios con el fin de evitar posibles rupturas en la cadena de suministro, ya que puede establecerse de manera certera el punto de pedido en el que es necesario emitir una orden de compra. 

Cabe resaltar que un exceso en los niveles de inventario es negativo para negocios con espacios limitados lo que puede conllevar al inadecuado uso de algunos productos con fecha cercana a la caducidad como medida de aprovechamiento de la materia prima, esto disminuye el nivel del servicio y por ende la competitividad del negocio.

En consecuencia, favorece la gestión de compras, evitando la adquisición urgente de materiales específicos necesarios para la elaboración de platos, por lo que asegura la disponibilidad constante de insumos para la totalidad del menú y garantiza la integridad de la oferta gastronómica, que es la principal fuente de ingresos del usuario. Evita además cambios abruptos en los stocks que podrían afectar el nivel de servicio como elemento esencial y diferenciador de la competencia y que además conforman una parte importante de los lineamientos estratégicos de la empresa para que su productividad sea sostenible.  

Dada la inexistencia de un sistema de gestión de stocks con un modelo predictivo en base a la información de ventas, el presente trabajo es una herramienta innovadora en el área gastronómica para aquellos negocios que desean aumentar su capacidad de  producción a través del manejo eficiente de su reabastecimiento.

Por otro lado, uno de los factores más importantes del costo de las operaciones empresariales está relacionado con el capital  invertido en la reposición de inventario necesario para el funcionamiento del local, el cual al quedar cautivo anula cualquier posibilidad de reinversión directa o de un nuevo destino generador de rentabilidad y posicionamiento.

En esta medida, dada la inmovilidad del capital, es determinante tomar decisiones que reduzcan costos, con lo cual surge la necesidad de regular los niveles de stock de los productos, cuantificarlos y gestionar su reposición. Aunque los inventarios almacenan valor, pueden generar una pérdida irrecuperable para el usuario cuando hay sobreabastecimiento, todo esto debido a que absorben el capital que podría estar disponible para otras opciones de inversión, sin embargo, lo paralizan y hasta pueden generar pérdidas cuando los consumibles se vencen o deterioran. También es importante la administración del espacio, ya que un stock superior al que puede ser preservado o refrigerado puede deteriorar sus componentes y por ende se podría ver afectada la calidad del producto final.

En otro sentido, disponer de pocos insumos genera una pausa considerable en la producción y  desmejora la calidad del producto ofrecido en el menú, lo que podría traducirse en el aumento de los costos por la carencia de artículos en el momento de ser demandados, pérdida de los beneficios originados de la venta imposible y el deterioro de la imagen comercial de la empresa, impulsando finalmente al consumidor a adquirir el bien a través de la competencia. Esta negativa también está asociada con la demora a la hora de satisfacer la demanda de pedidos al  momento en que son solicitados. 

Por otro lado, debe considerarse el análisis de la ventas como un factor fundamental para determinar la logística que deberá emplear la empresa para la reposición de inventario y así establecer su modelo de gestión. La predictibilidad entonces estará determinada por elementos probables y no por factores aleatorios, ya que la toma de decisiones se hará en función de estadísticas fidedignas arrojadas por el sistema alimentado con datos actualizados proporcionados por el usuario.

En consecuencia, la solución propuesta en esta tesis sobre la optimización de los niveles de stock en el rubro gastronómico de Argentina en el año 2025, utilizando modelos de Machine Learning para predecir los niveles requeridos en base a la información de ventas, resulta técnicamente factible y presenta un concepto innovador en gestión empresarial ya que pretende optimizar mecanismos de producción a través de la predictibilidad. Se configura en una herramienta avanzada para los propietarios de restaurantes y negocios afines, quienes a través de su implementación podrán obtener notables beneficios en cuanto a productividad y rentabilidad. Por último, el presente trabajo representa la oportunidad de crear un nuevo nicho de mercado en Argentina, específicamente la Ciudad Autónoma de Buenos Aires, basado en la innovación tecnológica por medio de la toma de decisiones informadas sustentadas en estadísticas proporcionadas por el mismo usuario.  


\subsection{User Research}\label{sec:user-research}

\subsubsection{Entrevistas}\label{sec:entrevistas}

Durante la etapa de descubrimiento del proyecto se consideró fundamental realizar entrevistas cualitativas con el fin de comprender a profundidad las múltiples dificultades que se enfrentan actualmente en la administración de negocios gastronómicos, buscando indagar tanto en las dinámicas como en los patrones diarios dentro del sector, especialmente en la gestión de inventarios y su impacto sobre dichos negocios.

Primeramente se entrevistó a Ulises Litterio, dueño de La Brava Burguer, la cual es una cadena gastronómica que consta de tres locales y 25 empleados. Se centra en la venta de hamburguesas y llevan a cabo todas las actividades correspondientes para el mantenimiento y desarrollo de calidad de las mismas que van desde la compra de materias primas hasta la venta final de estas con delivery propio. Ahora bien, el control del inventario se encuentra descentralizado, básicamente cada jefe de cocina realiza un conteo el cual se le informa al encargado, quien lo carga en una planilla de Excel que luego es revisada por el dueño (Ulises Litterio). Además, todas las compras necesarias son realizadas de manera semanal, aunque pueden repetirse en distintos momentos de la semana si hay productos o ingredientes faltantes.

El cálculo de reposición de stock se basa en datos como las ventas semanas anteriores, fechas clave, como feriados, y la intuición de los trabajadores sobre la demanda esperada. Sin embargo, gestionar el inventario de esta manera conlleva una alta tasa de error tanto por sobrestock como por faltantes, lo cual genera desperdicios tanto a nivel económico como alimenticio. Además, Ulises destaca la dificultad de consolidar datos de ventas debido a que operan con múltiples plataformas de delivery como Rappi y PedidosYa, cada una con su propio sistema individual, y aunque ha considerado la opción de emplear un software como MaxiRest, ha terminado por descartar dicha idea debido a los costos elevados que tendría que pagar.

En segundo lugar, se entrevistó a Oscar Campione, dueño de una rotisería familiar que opera con un único local instalado en una parte de la casa. La venta del negocio se basa única y exclusivamente en por delivery propio a través de PedidosYa. Además el personal se encuentra compuesto por cuatro personas los cuales son todos integrantes de la familia, exceptuando al repartidor.

Su modalidad de trabajo actualmente consiste en preparar y vender productos del mismo día, llevando manualmente un control de las mismas al finalizar la noche, de tal manera que el cálculo de reposición se hace constantemente para a la mañana siguiente reponer solamente lo vendido. Esta modalidad los expone a constantes errores que pueden afectar significativamente los costos para un local de dicho tamaño, ya sea por falta de planificación, sobreestimación, pérdida de productos, ventas, falta de disponibilidad o exceso de producto perecederos, lo cual a su vez genera en la familia frustración y pérdida de fidelidad.

Cabe resaltar que los dueños intentaron incorporar un software llamado “Delivery 5.0”. Sin embargo, este únicamente les permitía gestionar pedidos, dejando su mayor necesidad sin satisfacer, la predicción de ventas y reposición de stock. Por otro lado, el entrevistado señala que un software que pueda ayudarles a visualizar cuáles productos tienen mayor venta y cuánto se estima que vendieron en cierta cantidad de tiempo los ayudaría tanto a evitar el sobrestock como a generar promociones con base en datos reales.

\subsubsection{Encuesta}\label{sec:encuesta}

Con el objetivo de comprender los hábitos de consumo en locales gastronómicos y la percepción de los clientes respecto a la disponibilidad de los productos, se realizó una encuesta a más de 150 personas. El propósito de la misma fue poder identificar cómo la falta de stock y la planificación de inventario influyen directamente en la satisfacción y fidelización de los consumidores, un factor clave en el mundo gastronómico.

En cuanto a la frecuencia de consumo, el 69.8\% de los encuestados afirmó visitar locales gastronómicos o pedir por delivery al menos una vez por semana, lo que refleja una relación constante con este tipo de establecimientos.

Respecto a la disponibilidad de productos, el 47.8\%, es decir casi la mitad de los entrevistados señaló que en el último mes le sucedió que el plato o producto deseado no se encontraba disponible. Ahora bien, la falta de stock recurrente sí reflejó tener un impacto notable en la percepción del cliente ya que el 86,8\% indicó que reduciría sus visitas si un local no mantiene la disponibilidad de los platos que ofrece. Además, mientras que el 30,2\% expresó que esta situación le genera desconfianza y afecta negativamente su experiencia, el 62,92\% afirmó que le molesta al menos un poco cuando un plato no está disponible. Si se consideran ambos grupos, puede observarse que más del 93\% de los encuestados experimenta algún grado de molestia o descontento frente a la falta de disponibilidad, lo que evidencia el peso crítico de este factor en la satisfacción del cliente.

Por otra parte, el rol de la planificación de compras resultó clave: el 98,7\% de los participantes coincidió en que una mejor gestión del inventario puede mejorar significativamente el servicio. En la misma línea, el 99,4\% declaró que estaría más dispuesto a regresar a un local que siempre mantenga su menú disponible y con calidad constante, y el 98,7\% lo recomendaría más a terceros.

Finalmente, la encuesta mostró que la falta frecuente de platos afecta directamente la reputación de los locales (71,7\% lo considera un factor muy relevante) y entre las respuestas abiertas, los clientes destacaron como principal fuente de satisfacción la combinación de calidad, disponibilidad, buena atención y relación precio-calidad, confirmando la importancia de un sistema que permita optimizar el control de insumos.

\subsubsection{Conclusión}\label{sec:sintesis-user-research}

A partir del análisis hecho durante la presente etapa de investigación, puede observarse que tanto dueños como clientes coinciden en señalar la gestión de inventarios como un factor decisivo para la sostenibilidad de un local gastronómico. 

Desde la perspectiva de los propietarios de los restaurantes, todo lo que respecta al control manual y descentralizado del stock conduce a errores frecuentes que derivan en sobrecostos, desperdicio de productos y una pérdida de eficiencia operativa. Igualmente, la carencia de integración entre plataformas de venta y la ausencia de herramientas accesibles de predicción de demanda limitan la capacidad de anticipación y planificación, generando frustración y altos niveles de incertidumbre.

De la misma manera, los clientes expresaron el impacto negativo que la falta de disponibilidad tiene en su experiencia. Los resultados muestran que la fidelización y recomendación de un local dependen en gran medida de que los platos anunciados estén siempre disponibles y mantengan una calidad constante.

En conclusión, mientras que los dueños carecen de herramientas que les permitan planificar con precisión, los clientes demandan consistencia y confiabilidad en la oferta gastronómica. Este punto de encuentro fundamenta la necesidad de un sistema como SmartStocker, capaz de poder predecir ventas y a su vez optimizar el inventario, siendo así capaz de reducir desperdicios y además, asegurar la satisfacción del consumidor.

	\chapter{Estado del Arte}\label{chapter03}

Según \parencite{creswell2014}, el estado del arte constituye una instancia fundamental dentro del proceso de investigación debido a que permite conocer las soluciones vigentes a la problemática abordada, identificar los aportes realizados por otros actores y a su vez delimitar las distintas oportunidades de innovación para el proyecto en curso.

En este sentido, el presente apartado desarrolla un análisis del estado del arte en relación con el diseño de soluciones tecnológicas orientadas a la gestión inteligente del stock en el sector gastronómico, haciendo énfasis en aplicaciones que incorporan técnicas de \emph{Machine Learning} para la predicción de la demanda.

Ahora bien, con el objetivo de identificar los aportes diferenciales de la solución propuesta, se abordan tres tópicos esenciales: primeramente aplicaciones comerciales actuales, en segundo lugar proyectos académicos o experimentales relacionados, y por último una lectura estratégica basada en la teoría del Océano Azul para identificar oportunidades de innovación y creación de valor en espacios de mercado poco explorados.

\section{Soluciones tecnológicas con enfoque predictivo en el sector gastronómico internacional}\label{sec:estado-internacional}

A nivel mundial, la búsqueda por una gestión eficiente de los establecimientos gastronómicos ha impulsado el desarrollo de herramientas que integran inteligencia artificial, especialmente técnicas de Machine Learning. Estas soluciones han surgido en distintos países con el objetivo de anticipar la demanda y optimizar la administración de insumos. A continuación, se presentan dos de las plataformas más relevantes, con el fin de profundizar en el uso de la inteligencia artificial en el sector gastronómico a nivel global.

\begin{itemize}
    \item \textbf{5-Out}. Plataforma estadounidense que utiliza \emph{Machine Learning} para predecir ventas con integración de fuentes internas (TPV, programación de personal, reservas) y externas (clima, eventos), habilitando recomendaciones de compra para equilibrar inventario y reducir desperdicio. Es integrable con sistemas como Toast, Square y 7Shifts; además, anunció una alianza con Craftable para ampliar funciones administrativas y financieras \parencite{fiveout2025partnership}.

        Gracias al análisis predictivo, esta herramienta puede indicarle de forma detallada al sitio gastronómico la cantidad de productos que deberá comprar para mantener un equilibrio en la gestión del inventario. Esto ayuda a reducir significativamente el desperdicio de alimentos y aumenta en consecuencia la rentabilidad. 

        Además, esta aplicación es compatible e integrable a sistemas actuales como Toast, Square y 7Shifts, y consideran unirse a Craftable para sumar funciones administrativas y financieras a sus capacidades predictivas como alternativa innovadora en la gestión de restaurantes. 

    \item \textbf{Restoke}. Es una startup australiana que automatiza procesos administrativos en restaurantes mediante IA: control de inventario y compras, programación y proyección de ventas, e integración con POS y software contable. En 2024 fue noticia por su crecimiento y una ronda de financiación de USD 5,1 millones, con casos reportados de fuertes reducciones de costos operativos \parencite{santoreneos2024restoke}.
\end{itemize}

\subsection{Soluciones comerciales actuales en el sector gastronómico argentino}\label{sec:estado-arg}

Actualmente, en Argentina, el mercado del software especializado en el sector gastronómico se encuentra en constante crecimiento, que si bien todavía no ha explotado el mundo del aprendizaje automatizado, dispone de herramientas que facilitan la administración y operación diaria de distintos establecimientos como restaurantes, sitios de comida rápida, cafés y lugares afines. Es por esto que se resalta algunas de las plataformas más utilizadas dentro del presente rubro, Maxirest y Restô , ambas con un enfoque integral en la gestión del negocio.

\begin{itemize}
    \item \textbf{Maxirest} \parencite{latam2024maxirest}. Se trata de una herramienta reconocida en el mercado por su robustez y funcionalidad. Su enfoque se centra exclusivamente en el análisis descriptivo de datos históricos y es una de las soluciones más consolidadas del país en el ámbito gastronómico. La plataforma permite automatizar diversas tareas operativas, tales como la gestión de cocina, la toma de pedidos, la administración de reservas e inventario, y la generación automática de reportes. Además, incluye un módulo específico para delivery y takeaway, lo que resulta especialmente relevante considerando que el 47,6\% de las ventas del sector se realizan a través de estas dos modalidades.
    
    \item \textbf{Restô} \parencite{iprofesional2014resto}. Es un sistema de gestión gastronómica que forma parte del ecosistema del software Tango, el cual fue desarrollado por la empresa argentina Axoft. El propósito de este es brindar una solución a la necesidad de una herramienta tecnológica que simplifique los procesos del día a día de dicho sector. En línea con dicho objetivo, este brinda una administración integral en múltiples establecimientos gastronómicos con diversas funciones tales como la gestión de mesas y reservas, el control de stock de insumos, la facturación electrónica, el seguimiento de ventas, y muchas otras más. De igual manera, gracias a su enfoque modular y adaptable, es utilizado tanto por pequeños comercios como por grandes cadenas gastronómicas, lo que le ha permitido posicionarse como una de las soluciones locales más completas del mercado para la administración de negocios del rubro alimenticio.
\end{itemize}

Ahora bien, con el objetivo de sintetizar las principales características de las soluciones comerciales vigentes en Argentina y contrastarlas con la propuesta de SmartStocker, se elaboró la siguiente tabla comparativa. En ella se ponen en relación funcionalidades clave vinculadas a la gestión del inventario, los reportes en tiempo real y, especialmente, la incorporación de modelos de Machine Learning.  
\begin{table}[htbp]
  \centering
  \begin{tabular}{lccc}
    \hline
    \textbf{Funcionalidad} & \textbf{Maxirest} & \textbf{Restô} & \textbf{SmartStocker} \\
    \hline
    Reportes en tiempo real        & \cmark & \cmark & \cmark \\
    Control de inventario          & \cmark & \cmark & \cmark \\
    Integración con plataformas    & \cmark & \cmark & \cmark \\
    Facturación/gestión admin.     & \cmark & \cmark & \xmark \\
    Predicción de demanda (ML)     & \xmark & \xmark & \cmark \\
    Estimación dinámica de stock   & \xmark & \xmark & \cmark \\
    Variables externas en modelos  & \xmark & \xmark & \cmark \\
    Feedback de usuarios (ajuste)  & \xmark & \xmark & \cmark \\
    Tablero de métricas predictivas& \xmark & \xmark & \cmark \\
    Alertas preventivas de stock   & \xmark & \xmark & \cmark \\
    Enfoque exclusivo en gastronomía & \cmark & \cmark & \cmark \\
    \hline
  \end{tabular}
    \caption{Comparación de funcionalidades entre la competencia y SmartStocker.}
    \label{tab:comparacion-funcionalidades}
\end{table}
Tal como se observa, las plataformas locales se limitan a la automatización y análisis descriptivo, mientras que SmartStocker incorpora un enfoque predictivo con modelos de Machine Learning, evidenciando así el diferencial innovador del proyecto.

\subsection{Propuestas académicas y experimentales}\label{sec:academico}

Aunque el modelo predictivo se ha desarrollado en investigaciones de múltiples áreas, existen escasos trabajos académicos enfocados específicamente en el rubro gastronómico. Uno de ellos es \parencite{hari2024predictiwaste}: \emph{PredictiWaste: An ML-Powered Framework for Sustainable Food Inventory Optimization in Restaurants (marco de trabajo basado en aprendizaje automático para la optimización sostenible del inventario de alimentos en restaurantes)}, se introduce un modelo de análisis predictivo para reducir el desperdicio de alimentos en el inventario y la planificación de menús de restaurantes. 

Por consiguiente, realiza el pronóstico del desperdicio basándose en factores como el tipo y cantidad de alimento, número de comensales, tipo de evento, condiciones de almacenamiento, historial de compras, métodos de preparación, estacionalidad, ubicación y precio. El sistema clasifica el desecho como mínimo, moderado o alto y sugiere medidas de control de inventario, ayudando en consecuencia a las empresas a optimizar sus recursos y a promover la sostenibilidad. 

Igualmente, una de las propuestas más representativas se encuentra en \parencite{schmidt2022mlsales}, con el pronóstico de ventas de restaurantes basado en aprendizaje automático. Este estudio afirma que para la correcta gestión del personal en los restaurantes, es necesaria la previsión precisa de las ventas, y propone un caso práctico sobre diversos modelos de aprendizaje automático utilizando data real de las ventas de un negocio, tipo restaurante de tamaño medio, y la inclusión de modelos de redes neuronales recurrentes de tendencia para la comparación de su rendimiento a través de diversos métodos.

De la misma manera, otros estudios como el \parencite{tanizaki2019forecasting} sobre la previsión de la demanda en restaurantes mediante aprendizaje automático y análisis estadístico, construyen un modelo específico para cada tienda física que combina funcionalmente datos de diversos factores externos tales como ubicación, clima y eventos, mediante aprendizaje automático, para luego analizar el resultado de verificación con datos de otros negocios similares.

\subsection{Océano Azul}

La Estrategia del Océano Azul es una teoría innovadora conceptualizada en \parencite{kim2015estrategia} que redefine la competencia empresarial a través de la innovación, como creadora de oportunidades y crecimiento rentable en mercados inexplorados.

Según sus autores, los océanos rojos representan a todas las industrias que hoy operan en el mercado, donde sus características están altamente definidas y son aceptadas, lo que conlleva a la lucha entre empresas rivales por obtener el posicionamiento ventajoso de llevarse una mayor participación en la demanda existente. Es decir, es un nivel de subsistencia toda vez que al saturarse el mercado se reducen las perspectivas de rentabilidad y sostenibilidad, lo que ocasiona que los productos pierdan su rasgo de distinción y por ende se conviertan en genéricos.

Por el contrario, los océanos azules representan nichos de mercado no explorados donde es factible la creación de demanda con un crecimiento altamente rentable. En el contexto de los océanos azules, la competencia carece de relevancia, dado que los parámetros operativos aún no han sido definidos.

Los teóricos con la intención de cuantificar el impacto de la creación de los océanos azules sobre el crecimiento de una compañía, analizaron el lanzamiento de 108 negocios nuevos, lo que arrojó como resultado que el 86\% eran extensiones de líneas existentes, es decir, pertenecientes al océano rojo y el 14\% restante de los lanzamientos pertenecientes a océanos azules.

Esto subraya la diferencia entre la creación de nuevos mercados sin competencia que rompen el equilibrio tradicional entre valor y costos, generando crecimiento empresarial con ganancias más rentables y a su vez con la competencia tradicional directa en mercados ya existentes que se encuentran saturados por parte de empresas que luchan por el control de la demanda.

En la estrategia del océano azul, se definen como varias herramientas fundamentales a saber, la matriz de \emph{Eliminar-Reducir-Aumentar-Crear} como recurso central, que \guillemotleft obliga a las compañías no solo a perseguir simultáneamente la diferenciación y el bajo costo, sino también a romper con el compromiso de valor-costo. No se trata de crear una nueva curva de valor por crearla, sino de identificar las variables correctas que transformarán la dinámica de la industria.\guillemotright \parencite{kim2015estrategia}.

Este enfoque analítico desafía la lógica de la competencia tradicional de la industria brindando apertura a nuevos paradigmas de análisis económico, ya que impulsa a las empresas a pensar de manera diferente sobre cómo crear valor para el cliente. 

Ahora bien, la Matriz \emph{Eliminar-Reducir-Aumentar-Crear}, es la representación tabular del Marco de las Cuatro Acciones, y tiene como finalidad ayudar en la formación de las ideas para que la toma de decisiones tenga como objetivo la diferenciación y el bajo costo de manera integrada.

Asimismo, otra herramienta fundamental es la \emph{Curva de Valor}, la cual permite una visualización gráfica del perfil estratégico de una empresa en relación con los factores clave de la industria. Esta curva permite identificar si una propuesta se encuentra alejada o no, de lo convencional, permitiendo así la detección temprana de oportunidades de innovación que generen una ventaja competitiva sostenible dentro del mercado.


\subsection{Matriz ERAC}\label{sec:erac}

\begin{figure}[htbp]
    \centering
    \includegraphics[width=0.7\textwidth]{images/matrizEric.jpeg}
    \caption{Matriz ERAC de SmartStocker.}
    {\textit{Fuente: Elaboración propia.}}
    \label{fig:eric}
\end{figure}

Las variables identificadas en la matriz ERAC (Figura~\ref{fig:eric}) se utilizan como insumo para construir la Curva de Valor (Figura~\ref{fig:curva}), con el objetivo de visualizar el posicionamiento de \emph{SmartStocker} en relación con sus competidores en el mercado.

\FloatBarrier

\clearpage
\subsection{Curva de Valor}\label{sec:curva-valor}

\begin{figure}[htbp]
    \centering
    \includegraphics[width=0.7\textwidth]{images/curvaValor.jpeg}
    \caption{Curva de valor.}
    {\textit{Fuente: Elaboración propia.}}
    \label{fig:curva}
\end{figure}

\FloatBarrier

\subsection{Conclusión}\label{sec:sintesis-estado}

Una vez realizado el análisis del estado del arte y de las herramientas que utiliza, se evidencia el diferencial del presente proyecto ya que no existen en la actualidad soluciones tecnológicas predictivas en el mercado nacional que conduzcan a una precisa gestión del inventario en el rubro gastronómico en base a la información de ventas, toda vez que las otras plataformas en el mercado se limitan a funciones descriptivas, basadas en estadísticas generales de consumo.

Asimismo, la presente tesis desarrolla un enfoque novedoso ya que utiliza el modelo de Machine Learning para predecir las cantidades de producto necesarias para el funcionamiento óptimo del usuario, ajustar los niveles de existencias en tiempo real y mantener los niveles de stock necesarios para que el local sea altamente productivo y su rentabilidad sea sostenible.

Mantener el inventario en un punto de equilibrio óptimo impacta directamente en la eficiencia operativa del negocio ya que implica la reducción de costos y la posibilidad de minimizar el capital inactivo en productos consumibles. Como consecuencia, genera grandes beneficios tales como el control de los desperdicios que puede traducirse a corto plazo en una reducción de gastos. Permite la rotación correcta de los insumos, evita la sobrecompra y la posible merma por caducidad ya que los alimentos tienen un tiempo de vida útil limitado, y luego de transcurrido este, no queda otra que descartarlos por control sanitario. Consecuencialmente, también se evita el gasto colateral que se genera en tiempo horas-hombre en personal utilizado para eliminar desechos, asear y reacondicionar el lugar.

Además, permite determinar el stock de seguridad, lo que es igual a la cantidad mínima de insumos necesarios para la producción base del negocio, facilitando así la gestión de mantenimiento de los volúmenes necesarios con el fin de evitar posibles rupturas en la cadena de suministro, ya que puede establecerse de manera certera el punto de pedido en el que es necesario emitir una orden de compra. 

Cabe resaltar que un exceso en los niveles de inventario es negativo para negocios con espacios limitados lo que puede conllevar al inadecuado uso de algunos productos con fecha cercana a la caducidad como medida de aprovechamiento de la materia prima, esto disminuye el nivel del servicio y por ende la competitividad del negocio.

En consecuencia, favorece la gestión de compras, evitando la adquisición urgente de materiales específicos necesarios para la elaboración de platos, por lo que asegura la disponibilidad constante de insumos para la totalidad del menú y garantiza la integridad de la oferta gastronómica, que es la principal fuente de ingresos del usuario. Evita además cambios abruptos en los stocks que podrían afectar el nivel de servicio como elemento esencial y diferenciador de la competencia y que además conforman una parte importante de los lineamientos estratégicos de la empresa para que su productividad sea sostenible.  

Dada la inexistencia de un sistema de gestión de stocks con un modelo predictivo en base a la información de ventas, el presente trabajo es una herramienta innovadora en el área gastronómica para aquellos negocios que desean aumentar su capacidad de  producción a través del manejo eficiente de su reabastecimiento.

Por otro lado, uno de los factores más importantes del costo de las operaciones empresariales está relacionado con el capital  invertido en la reposición de inventario necesario para el funcionamiento del local, el cual al quedar cautivo anula cualquier posibilidad de reinversión directa o de un nuevo destino generador de rentabilidad y posicionamiento.

En esta medida, dada la inmovilidad del capital, es determinante tomar decisiones que reduzcan costos, con lo cual surge la necesidad de regular los niveles de stock de los productos, cuantificarlos y gestionar su reposición. Aunque los inventarios almacenan valor, pueden generar una pérdida irrecuperable para el usuario cuando hay sobreabastecimiento, todo esto debido a que absorben el capital que podría estar disponible para otras opciones de inversión, sin embargo, lo paralizan y hasta pueden generar pérdidas cuando los consumibles se vencen o deterioran. También es importante la administración del espacio, ya que un stock superior al que puede ser preservado o refrigerado puede deteriorar sus componentes y por ende se podría ver afectada la calidad del producto final.

En otro sentido, disponer de pocos insumos genera una pausa considerable en la producción y  desmejora la calidad del producto ofrecido en el menú, lo que podría traducirse en el aumento de los costos por la carencia de artículos en el momento de ser demandados, pérdida de los beneficios originados de la venta imposible y el deterioro de la imagen comercial de la empresa, impulsando finalmente al consumidor a adquirir el bien a través de la competencia. Esta negativa también está asociada con la demora a la hora de satisfacer la demanda de pedidos al  momento en que son solicitados. 

Por otro lado, debe considerarse el análisis de la ventas como un factor fundamental para determinar la logística que deberá emplear la empresa para la reposición de inventario y así establecer su modelo de gestión. La predictibilidad entonces estará determinada por elementos probables y no por factores aleatorios, ya que la toma de decisiones se hará en función de estadísticas fidedignas arrojadas por el sistema alimentado con datos actualizados proporcionados por el usuario.

En consecuencia, la solución propuesta en esta tesis sobre la optimización de los niveles de stock en el rubro gastronómico de Argentina en el año 2025, utilizando modelos de Machine Learning para predecir los niveles requeridos en base a la información de ventas, resulta técnicamente factible y presenta un concepto innovador en gestión empresarial ya que pretende optimizar mecanismos de producción a través de la predictibilidad. Se configura en una herramienta avanzada para los propietarios de restaurantes y negocios afines, quienes a través de su implementación podrán obtener notables beneficios en cuanto a productividad y rentabilidad. Por último, el presente trabajo representa la oportunidad de crear un nuevo nicho de mercado en Argentina, específicamente la Ciudad Autónoma de Buenos Aires, basado en la innovación tecnológica por medio de la toma de decisiones informadas sustentadas en estadísticas proporcionadas por el mismo usuario.  


\section{User research}\label{sec:user-research}

\subsection{Entrevistas}\label{sec:entrevistas}

Durante la etapa de descubrimiento del proyecto se consideró fundamental realizar entrevistas cualitativas con el fin de comprender a profundidad las múltiples dificultades que se enfrentan actualmente en la administración de negocios gastronómicos, buscando indagar tanto en las dinámicas como en los patrones diarios dentro del sector, especialmente en la gestión de inventarios y su impacto sobre dichos negocios.

Primeramente se entrevistó a Ulises Litterio, dueño de La Brava Burguer, la cual es una cadena gastronómica que consta de tres locales y 25 empleados. Se centra en la venta de hamburguesas y llevan a cabo todas las actividades correspondientes para el mantenimiento y desarrollo de calidad de las mismas que van desde la compra de materias primas hasta la venta final de estas con delivery propio. Ahora bien, el control del inventario se encuentra descentralizado, básicamente cada jefe de cocina realiza un conteo el cual se le informa al encargado, quien lo carga en una planilla de Excel que luego es revisada por el dueño (Ulises Litterio). Además, todas las compras necesarias son realizadas de manera semanal, aunque pueden repetirse en distintos momentos de la semana si hay productos o ingredientes faltantes.

El cálculo de reposición de stock se basa en datos como las ventas semanas anteriores, fechas clave, como feriados, y la intuición de los trabajadores sobre la demanda esperada. Sin embargo, gestionar el inventario de esta manera conlleva una alta tasa de error tanto por sobrestock como por faltantes, lo cual genera desperdicios tanto a nivel económico como alimenticio. Además, Ulises destaca la dificultad de consolidar datos de ventas debido a que operan con múltiples plataformas de delivery como Rappi y PedidosYa, cada una con su propio sistema individual, y aunque ha considerado la opción de emplear un software como MaxiRest, ha terminado por descartar dicha idea debido a los costos elevados que tendría que pagar.

En segundo lugar, se entrevistó a Oscar Campione, dueño de una rotisería familiar que opera con un único local instalado en una parte de la casa. La venta del negocio se basa única y exclusivamente en por delivery propio a través de PedidosYa. Además el personal se encuentra compuesto por cuatro personas los cuales son todos integrantes de la familia, exceptuando al repartidor.

Su modalidad de trabajo actualmente consiste en preparar y vender productos del mismo día, llevando manualmente un control de las mismas al finalizar la noche, de tal manera que el cálculo de reposición se hace constantemente para a la mañana siguiente reponer solamente lo vendido. Esta modalidad los expone a constantes errores que pueden afectar significativamente los costos para un local de dicho tamaño, ya sea por falta de planificación, sobreestimación, pérdida de productos, ventas, falta de disponibilidad o exceso de producto perecederos, lo cual a su vez genera en la familia frustración y pérdida de fidelidad.

Cabe resaltar que los dueños intentaron incorporar un software llamado “Delivery 5.0”. Sin embargo, este únicamente les permitía gestionar pedidos, dejando su mayor necesidad sin satisfacer, la predicción de ventas y reposición de stock. Por otro lado, el entrevistado señala que un software que pueda ayudarles a visualizar cuáles productos tienen mayor venta y cuánto se estima que vendieron en cierta cantidad de tiempo los ayudaría tanto a evitar el sobrestock como a generar promociones con base en datos reales.

\subsection{Encuesta}\label{sec:encuesta}

Con el objetivo de comprender los hábitos de consumo en locales gastronómicos y la percepción de los clientes respecto a la disponibilidad de los productos, se realizó una encuesta a más de 150 personas. El propósito de la misma fue poder identificar cómo la falta de stock y la planificación de inventario influyen directamente en la satisfacción y fidelización de los consumidores, un factor clave en el mundo gastronómico.

En cuanto a la frecuencia de consumo, el 69.8\% de los encuestados afirmó visitar locales gastronómicos o pedir por delivery al menos una vez por semana, lo que refleja una relación constante con este tipo de establecimientos.

Respecto a la disponibilidad de productos, el 47.8\%, es decir casi la mitad de los entrevistados señaló que en el último mes le sucedió que el plato o producto deseado no se encontraba disponible. Ahora bien, la falta de stock recurrente sí reflejó tener un impacto notable en la percepción del cliente ya que el 86,8\% indicó que reduciría sus visitas si un local no mantiene la disponibilidad de los platos que ofrece. Además, mientras que el 30,2\% expresó que esta situación le genera desconfianza y afecta negativamente su experiencia, el 62,92\% afirmó que le molesta al menos un poco cuando un plato no está disponible. Si se consideran ambos grupos, puede observarse que más del 93\% de los encuestados experimenta algún grado de molestia o descontento frente a la falta de disponibilidad, lo que evidencia el peso crítico de este factor en la satisfacción del cliente.

Por otra parte, el rol de la planificación de compras resultó clave: el 98,7\% de los participantes coincidió en que una mejor gestión del inventario puede mejorar significativamente el servicio. En la misma línea, el 99,4\% declaró que estaría más dispuesto a regresar a un local que siempre mantenga su menú disponible y con calidad constante, y el 98,7\% lo recomendaría más a terceros.

Finalmente, la encuesta mostró que la falta frecuente de platos afecta directamente la reputación de los locales (71,7\% lo considera un factor muy relevante) y entre las respuestas abiertas, los clientes destacaron como principal fuente de satisfacción la combinación de calidad, disponibilidad, buena atención y relación precio-calidad, confirmando la importancia de un sistema que permita optimizar el control de insumos.

\subsection{Conclusión}\label{sec:sintesis-user-research}

A partir del análisis hecho durante la presente etapa de investigación, puede observarse que tanto dueños como clientes coinciden en señalar la gestión de inventarios como un factor decisivo para la sostenibilidad de un local gastronómico. 

Desde la perspectiva de los propietarios de los restaurantes, todo lo que respecta al control manual y descentralizado del stock conduce a errores frecuentes que derivan en sobrecostos, desperdicio de productos y una pérdida de eficiencia operativa. Igualmente, la carencia de integración entre plataformas de venta y la ausencia de herramientas accesibles de predicción de demanda limitan la capacidad de anticipación y planificación, generando frustración y altos niveles de incertidumbre.

De la misma manera, los clientes expresaron el impacto negativo que la falta de disponibilidad tiene en su experiencia. Los resultados muestran que la fidelización y recomendación de un local dependen en gran medida de que los platos anunciados estén siempre disponibles y mantengan una calidad constante.

En conclusión, mientras que los dueños carecen de herramientas que les permitan planificar con precisión, los clientes demandan consistencia y confiabilidad en la oferta gastronómica. Este punto de encuentro fundamenta la necesidad de un sistema como SmartStocker, capaz de poder predecir ventas y a su vez optimizar el inventario, siendo así capaz de reducir desperdicios y además, asegurar la satisfacción del consumidor.

	\chapter{Descripción}\label{chapter04}

El presente capítulo desarrolla la descripción integral de la solución propuesta, abordando sus fundamentos conceptuales, técnicos y operativos. Se expone la forma en que la plataforma materializa los objetivos definidos en las etapas de análisis, describiendo los principales componentes que la conforman y la lógica que guía su funcionamiento. Asimismo, se presenta la manera en que las decisiones de diseño, arquitectura y desarrollo se articulan para garantizar la eficiencia, la escalabilidad y la confiabilidad del sistema.

\section{Requerimientos}\label{sec:requerimientos}
A continuación se presentan los requerimientos de la solución, organizados en dos categorías: requerimientos funcionales y no funcionales. Los requerimientos funcionales representan las operaciones y servicios que la plataforma debe ofrecer al usuario final, mientras que los no funcionales establecen criterios de calidad que condicionan la forma en que el sistema debe operar \parencite{ieee2008}.

SmartStocker está diseñado para asistir a pequeños y medianos establecimientos gastronómicos en la gestión de su inventario. A partir de la integración con plataformas de ventas y el análisis de datos históricos mediante técnicas de Machine Learning, el sistema proyecta la demanda futura, calcula el stock necesario y genera alertas tempranas para evitar pérdidas por desabastecimiento o exceso. Asimismo, provee un tablero con métricas relevantes y permite al usuario interactuar con el modelo de predicción para ajustarlo a la realidad de su negocio mediante feedback.

\subsection{Requerimientos funcionales}\label{sec:requerimientos-funcionales}
Los requerimientos funcionales describen las acciones observables que el sistema debe llevar a cabo para satisfacer las necesidades del usuario \parencite{ieee2008}.
\begin{enumerate}[label=\textbf{RF\arabic*}, leftmargin=2.5cm]
    \item El sistema debe permitir el alta de cuentas de restaurantes, con autenticación segura mediante correo electrónico y contraseña.
    \item El sistema debe permitir registrar, modificar y eliminar productos gastronómicos, vinculados con los insumos que los componen.
    \item El sistema debe facilitar la administración de inventario, incluyendo actualización de cantidades disponibles y definición de umbrales mínimos de stock.
    \item El sistema debe conectarse con plataformas de delivery en CABA (por ejemplo, PedidosYa) para importar datos de ventas en tiempo real.
    \item El sistema debe permitir la carga manual de datos históricos de ventas a través de archivos en formato \texttt{.csv} o \texttt{.xlsx}.
    \item El sistema debe ejecutar predicciones de ventas que sean basadas en modelos de Machine Learning, considerando tanto información histórica como variables externas (clima, feriados, días de la semana).
    \item El sistema debe calcular automáticamente la cantidad de inventario recomendada según los resultados de las predicciones generadas.
    \item El sistema debe emitir alertas cuando un insumo se encuentre por debajo del nivel mínimo configurado por el usuario.
    \item El sistema debe permitir al usuario proporcionar retroalimentación respecto a las predicciones, incorporando estos datos para el ajuste del modelo.
\end{enumerate}

\subsection{Requerimientos no funcionales}\label{sec:requerimientos-no-funcionales}
Los requerimientos no funcionales especifican condiciones de calidad y restricciones técnicas que determinan cómo debe operar la solución, más allá de las funciones explícitas que ofrece \parencite{ieee2008}.
\begin{enumerate}[label=\textbf{RNF\arabic*}, leftmargin=2.8cm]
    \item La interfaz debe ser adaptable (responsive) para asegurar su correcto uso en computadoras de escritorio, tablets y dispositivos móviles.
    \item La aplicación debe garantizar disponibilidad continua, las veinticuatro horas del día, los siete días de la semana, para el acceso a predicciones, métricas e informes en cualquier momento.
    \item El sistema debe ser compatible con los navegadores más utilizados (Google Chrome, Mozilla Firefox, Microsoft Edge y Safari) en sus versiones estables recientes.
    \item La infraestructura debe poder escalar automáticamente para soportar un crecimiento sostenido de usuarios sin afectar el rendimiento.
    \item El sistema debe desplegarse en una plataforma en la nube (por ejemplo, AWS Amplify) que garantice al menos un 99\% de disponibilidad mensual.
\end{enumerate}

\section{Diagramas}\label{sec:diagramas}

Para comprender de manera clara la interacción entre los usuarios y el sistema, así como el flujo de los procesos internos, se emplean diferentes tipos de diagramas. Estas representaciones gráficas permiten comunicar de forma visual los requerimientos, las funcionalidades y la lógica de operación de la aplicación, favoreciendo la comprensión tanto de aspectos funcionales como de diseño \parencite{booch2005uml}. 

A continuación, se detallan los posibles flujos funcionales que se encuentran disponibles en SmartStocker.

\subsection{Diagramas de Casos de Uso}\label{sec:diagramas-casos-uso}

Un diagrama de caso de uso es una representación visual que muestra cómo los actores (usuarios u otros sistemas) interactúan con las funcionalidades principales de una aplicación. Estos diagramas permiten modelar el comportamiento esperado desde el punto de vista del usuario, identificando qué operaciones puede ejecutar y cómo se relacionan con el sistema \parencite{jacobson1992usecase}.
\begin{figure}[htbp]
    \centering
    \includegraphics[width=0.7\textwidth]{images/DiagramaCasosDeUsoTesis.png}
    \caption{Diagrama de Casos de Uso}
    {\textit{Fuente: Elaboración propia.}}
    \label{fig:casos-de-uso}
\end{figure}

\subsection{Diagramas de Flujo}\label{sec:diagramas-flujo}

Los diagramas de flujo o de procesos son herramientas visuales que representan de manera secuencial los pasos y decisiones que conforman un procedimiento. Su utilización permite clarificar la lógica del negocio, identificar posibles ineficiencias y redundancias, así como facilitar la comunicación entre los equipos técnicos y los usuarios. De esta manera, constituyen un recurso fundamental para comprender, analizar, documentar y mejorar procesos organizacionales \parencite{asq2025flowchart}.

\subsubsection{Registro y Autenticación}
Con el fin de garantizar la seguridad y la personalización de los datos, el acceso a la información y a las funcionalidades críticas de SmartStocker se encuentra restringido a usuarios autenticados. Para ello, la plataforma implementa un flujo de registro inicial y un mecanismo de inicio de sesión que permite identificar de manera única a cada restaurante o local gastronómico.

Durante la etapa de registro, el usuario completa un formulario en línea con los datos básicos del negocio.Estos datos son validados por el sistema para comprobar su integridad y formato. En caso de inconsistencias, el sistema informa los errores y solicita su corrección antes de continuar. Si la información ingresada es válida, SmartStocker redirige al usuario a la página de inicio y registra la solicitud correspondiente. Posteriormente, la empresa valida y aprueba el alta del negocio, enviando un correo electrónico con las credenciales definitivas de acceso y confirmando la activación de la cuenta en el entorno B2B. Este proceso garantiza que cada registro corresponda efectivamente a un establecimiento gastronómico verificado, asegurando así la integridad de la red y la trazabilidad de los datos administrados por la plataforma.
\begin{figure}[htbp]
    \centering
    \includegraphics[width=1\textwidth]{images/DiagramaRegistroDeUsuarioB2B.png}
    \caption{Flujo de Registro del Usuario.}
    {\textit{Fuente: Elaboración propia.}}
    \label{fig:flujo-registro}
\end{figure}

Posteriormente, mediante la autenticación, los usuarios acceden a su espacio de trabajo personalizado, desde el cual es posible consultar métricas, visualizar predicciones de demanda, administrar productos e ingredientes y recibir alertas relacionadas con el stock disponible. Este proceso de inicio de sesión no solo protege la integridad de la información, sino que también garantiza que las recomendaciones generadas por el sistema respondan a las características particulares de cada negocio gastronómico.

\begin{figure}[htbp]
    \centering
    \includegraphics[width=1\textwidth]{images/DiagramaInicioDeSesion.png}
    \caption{Flujo de Autenticación del Usuario.}
    {\textit{Fuente: Elaboración propia.}}
    \label{fig:flujo-autenticacion}
\end{figure}

\subsubsection{Generación de Predicciones}

La capacidad de anticipar la demanda de productos constituye la funcionalidad central de SmartStocker, ya que permite a los restaurantes y locales gastronómicos tomar decisiones fundamentadas sobre compras y reposiciones de insumos. En este proceso, el sistema aplica modelos de Machine Learning que, a partir de datos históricos y variables externas, generan estimaciones confiables de ventas futuras.
\begin{figure}[htbp]
    \centering
    \includegraphics[width=0.7\textwidth]{images/DiagramaDePrediccionTesis.png}
    \caption{Flujo de Predicción de Ventas.}
    {\textit{Fuente: Elaboración propia.}}
    \label{fig:flujo-prediccion}
\end{figure}

\section{Identidad de marca}\label{subsec:identidad-marca}
SmartStocker es una marca centrada en la toma de decisiones basada en datos para la gestión de inventarios gastronómicos. Su identidad se sostiene en tres ejes: claridad (información entendible y accionable), confiabilidad (resultados consistentes) y agilidad (respuestas oportunas para planificar compras y evitar quiebres o excedentes). La narrativa de marca comunica simplicidad y precisión: transformar registros de venta y contexto operativo en recomendaciones concretas de stock y alertas preventivas.

\subsection{Misión}\label{subsec:mision}
Impulsar la rentabilidad de restaurantes y locales gastronómicos mediante una plataforma que integra datos de venta en tiempo real, predice la demanda y traduce esos resultados en recomendaciones de compra y niveles de stock accionables. SmartStocker reduce quiebres y desperdicios, simplifica la operación diaria y convierte la gestión de inventario en un proceso preciso, automatizado y basado en evidencia.

\subsection{Visión}\label{subsec:vision}
Ser la referencia regional en optimización de inventarios gastronómicos mediante analítica predictiva, integrando datos operativos y conocimiento experto para elevar la toma de decisiones cotidiana.

\subsection{Nombre}\label{subsec:nombre}
El nombre SmartStocker combina \textit{Smart} (inteligente) y \textit{Stocker} (almacenista). La construcción comunica de manera directa el beneficio central: administrar inventarios con inteligencia. Es breve, fácil de pronunciar y recordar, y al estar en inglés facilita su adopción en distintos mercados y contextos tecnológicos.

\subsection{Identidad visual}\label{subsec:identidad-visual}
\paragraph{Concepto.} La identidad visual representa el pasaje de datos a decisiones. Se recurre a formas claras y proporciones estables que remiten a control, previsión y orden, evitando recursos recargados que resten legibilidad en entornos digitales.

\paragraph{Logotipo tipográfico.}
El logotipo tipográfico prioriza legibilidad y memorabilidad. La diferenciación visual se logra por color: \textit{Smart} en tono oscuro (negro/gris muy profundo) y \textit{Stocker} en azul corporativo, lo que facilita la lectura y refuerza el territorio semántico del producto (gestión de stock). Además, se utiliza capitalización inicial en cada palabra y composición en una sola línea.

\begin{figure}[htbp]
    \centering
    \includegraphics[width=0.80\textwidth]{images/smartstocker-logotipo-tipografico.png}
    \caption{Logotipo tipográfico de SmartStocker.}
    {\textit{Fuente: Elaboración propia.}}
    \label{fig:logo-tipografico-smartstocker}
\end{figure}

\paragraph{Imagotipo (isotipo + logotipo).}
El imagotipo integra un isotipo geométrico que sugiere \textit{contenedor/caja} (inventario) y una flecha ascendente (proyección/abastecimiento), combinado con el logotipo. El isotipo admite uso independiente en tamaños reducidos (favicon, icono de app), mientras que el imagotipo es la versión preferente para cabeceras, presentaciones y piezas institucionales.

\begin{figure}[htbp]
    \centering
    \includegraphics[width=0.70\textwidth]{images/smartstocker-imagotipo.png}
    \caption{Imagotipo de SmartStocker (composición isotipo + logotipo).}
    {\textit{Fuente: Elaboración propia.}}
    \label{fig:imagotipo-smartstocker}
\end{figure}

\subsection{Paleta de Colores}\label{subsec:paleta-colores}

El diseño visual de SmartStocker se sustenta en una paleta cromática la cual fue escogida con el propósito de transmitir profesionalismo, confianza y modernidad, atributos que resultan esenciales en un sistema orientado a la toma de decisiones estratégicas en el sector gastronómico. Ahora bien, la elección de los tonos se realizó considerando tanto la dimensión estética como la funcionalidad comunicativa de cada color dentro de la interfaz.

Los colores principales corresponden a una gama de azules, donde el azul corporativo (\#1E40AF) se utiliza en encabezados, botones primarios y elementos de navegación clave. Este tono se asocia tradicionalmente con la seriedad, la estabilidad y la confianza, lo que refuerza la credibilidad del sistema frente a los usuarios que deben basar sus decisiones en datos precisos. Como complemento, el azul claro (\#3B82F6) aparece en estados de interacción como hover y en componentes secundarios, aportando un contraste visual que mantiene la coherencia cromática y facilita la detección rápida de acciones disponibles.

En cuanto a los colores semánticos, se definieron tres que cumplen un rol fundamental en la comunicación de estados del sistema: verde éxito (\#10B981), naranja advertencia (\#F59E0B) y rojo crítico (\#EF4444). Estos colores no solo cumplen con la convención culturalmente reconocida (verde como positivo, rojo como error), sino que también permiten que el usuario identifique de forma inmediata la condición de un insumo o la validez de una acción.

Los colores de apoyo, grises en diferentes intensidades y blanco como fondo principal, cumplen la función de dar equilibrio visual. El gris oscuro (\#1F2937) se emplea en textos principales para asegurar legibilidad, mientras que el gris medio (\#6B7280) se reserva para información secundaria, evitando así la sobrecarga cognitiva. Además, el gris claro (\#F3F4F6) y el azul muy claro (\#EFF6FF) sirven como fondos sutiles y separadores, permitiendo estructurar la información en bloques diferenciados sin necesidad de líneas divisorias excesivas.

En conjunto, esta paleta de colores no responde únicamente a una intención estética, sino que está al servicio de la usabilidad, la accesibilidad y la experiencia de usuario. Cada tonalidad se integra en un sistema visual coherente que facilita la navegación, refuerza el reconocimiento de patrones y garantiza que los usuarios puedan interactuar con la plataforma de forma intuitiva y eficiente.

\begin{table}[htbp]
\centering
\caption{Paleta de colores empleada en SmartStocker.}
\label{tab:paleta-colores}

\renewcommand{\arraystretch}{1.25}
\begin{tabular}{|l|c|c|}
\hline
\textbf{Categoría} & \textbf{Color} & \textbf{Código Hex} \\ \hline
Azul corporativo & \cellcolor[HTML]{1E40AF} & \#1E40AF \\ \hline
Azul claro & \cellcolor[HTML]{3B82F6} & \#3B82F6 \\ \hline
Verde éxito & \cellcolor[HTML]{10B981} & \#10B981 \\ \hline
Naranja advertencia & \cellcolor[HTML]{F59E0B} & \#F59E0B \\ \hline
Rojo crítico & \cellcolor[HTML]{EF4444} & \#EF4444 \\ \hline
Gris oscuro (texto principal) & \cellcolor[HTML]{1F2937} & \#1F2937 \\ \hline
Gris medio (texto secundario) & \cellcolor[HTML]{6B7280} & \#6B7280 \\ \hline
Gris claro (fondos) & \cellcolor[HTML]{F3F4F6} & \#F3F4F6 \\ \hline
Azul muy claro (fondos) & \cellcolor[HTML]{EFF6FF} & \#EFF6FF \\ \hline
\end{tabular}

\vspace{0.3em}
\captionsetup{justification=centering}
{\small \textit{Fuente: Elaboración propia.}}
\end{table}

\subsection{Pantallas de la aplicación}\label{subsec:pantallas-aplicacion}
A continuación se presentan las principales pantallas de la aplicación SmartStocker, ordenadas según el flujo de interacción. Estas capturas abarcan: la experiencia inicial (Figuras~\ref{fig:ux-landing1} a \ref{fig:ux-landing5} y \ref{fig:ux-login}); el dashboard (Figuras~\ref{fig:ux-dashboard1} a \ref{fig:ux-dashboard3}); la gestión de datos (Figuras~\ref{fig:ux-ingredientes}, \ref{fig:ux-items}, \ref{fig:ux-nuevo-item} y \ref{fig:ux-nuevo-ingrediente}); y las funcionalidades de predicción de ventas (Figuras~\ref{fig:ux-predicciones} a \ref{fig:ux-prediccion2}).

\begin{figure}[htbp]
    \centering
    \includegraphics[width=0.9\textwidth]{images/landing1.png}
    \caption{Pantalla de inicio (Landing) – Parte 1.}
    {\textit{Fuente: Elaboración propia.}}
    \label{fig:ux-landing1}
\end{figure}

\begin{figure}[htbp]
    \centering
    \includegraphics[width=0.9\textwidth]{images/landing2.png}
    \caption{Pantalla de inicio (Landing) – Parte 2.}
    {\textit{Fuente: Elaboración propia.}}
    \label{fig:ux-landing2}
\end{figure}

\begin{figure}[htbp]
    \centering
    \includegraphics[width=0.9\textwidth]{images/landing3.png}
    \caption{Pantalla de inicio (Landing) – Parte 3.}
    {\textit{Fuente: Elaboración propia.}}
    \label{fig:ux-landing3}
\end{figure}

\begin{figure}[htbp]
    \centering
    \includegraphics[width=0.9\textwidth]{images/landing4.png}
    \caption{Pantalla de inicio (Landing) – Parte 4.}
    {\textit{Fuente: Elaboración propia.}}
    \label{fig:ux-landing4}
\end{figure}

\begin{figure}[htbp]
    \centering
    \includegraphics[width=0.9\textwidth]{images/landing5.png}
    \caption{Pantalla de inicio (Landing) – Parte 5.}
    {\textit{Fuente: Elaboración propia.}}
    \label{fig:ux-landing5}
\end{figure}

\begin{figure}[htbp]
    \centering
    \includegraphics[width=0.9\textwidth]{images/login.png}
    \caption{Pantalla de inicio de sesión.}
    {\textit{Fuente: Elaboración propia.}}
    \label{fig:ux-login}
\end{figure}

\begin{figure}[htbp]
    \centering
    \includegraphics[width=0.9\textwidth]{images/dashboard1.png}
    \caption{Dashboard – Vista general con métricas clave.}
    {\textit{Fuente: Elaboración propia.}}
    \label{fig:ux-dashboard1}
\end{figure}

\begin{figure}[htbp]
    \centering
    \includegraphics[width=0.9\textwidth]{images/dashboard2.png}
    \caption{Dashboard – Predicciones de demanda y niveles de stock sugeridos.}
    {\textit{Fuente: Elaboración propia.}}
    \label{fig:ux-dashboard2}
\end{figure}

\begin{figure}[htbp]
    \centering
    \includegraphics[width=0.9\textwidth]{images/dashboard3.png}
    \caption{Dashboard – Alertas y seguimiento de inventario por insumo.}
    {\textit{Fuente: Elaboración propia.}}
    \label{fig:ux-dashboard3}
\end{figure}

\begin{figure}[htbp]
    \centering
    \includegraphics[width=0.9\textwidth]{images/ingredientes.png}
    \caption{Pantalla de gestión de ingredientes.}
    {\textit{Fuente: Elaboración propia.}}
    \label{fig:ux-ingredientes}
\end{figure}

\begin{figure}[htbp]
    \centering
    \includegraphics[width=0.9\textwidth]{images/items.png}
    \caption{Pantalla de gestión de ítems.}
    {\textit{Fuente: Elaboración propia.}}
    \label{fig:ux-items}
\end{figure}

\begin{figure}[htbp]
    \centering
    \includegraphics[width=0.9\textwidth]{images/nuevoItem.png}
    \caption{Pantalla para agregar un nuevo ítem.}
    {\textit{Fuente: Elaboración propia.}}
    \label{fig:ux-nuevo-item}
\end{figure}

\begin{figure}[htbp]
    \centering
    \includegraphics[width=0.9\textwidth]{images/nuevoIngrediente.png}
    \caption{Pantalla para agregar un nuevo ingrediente.}
    {\textit{Fuente: Elaboración propia.}}
    \label{fig:ux-nuevo-ingrediente}
\end{figure}

\begin{figure}[htbp]
    \centering
    \includegraphics[width=0.9\textwidth]{images/predicciones.png}
    \caption{Pantalla consolidada de predicciones.}
    {\textit{Fuente: Elaboración propia.}}
    \label{fig:ux-predicciones}
\end{figure}

\begin{figure}[htbp]
    \centering
    \includegraphics[width=0.9\textwidth]{images/prediccion1.png}
    \caption{Resultado de predicciones.}
    {\textit{Fuente: Elaboración propia.}}
    \label{fig:ux-prediccion1}
\end{figure}

\begin{figure}[htbp]
    \centering
    \includegraphics[width=0.9\textwidth]{images/prediccion2.png}
    \caption{Ingredientes necesarios.}
    {\textit{Fuente: Elaboración propia.}}
    \label{fig:ux-prediccion2}
\end{figure}

\section{Funcionalidades}\label{sec:funcionalidades}

\subsection{Predicción de Ventas}\label{sec:prediccion-ventas}

En el sector gastronómico, la capacidad de anticipar la demanda se ha convertido en un factor estratégico para reducir pérdidas, optimizar el uso de insumos y garantizar la continuidad del servicio. Bajo dicha premisa, SmartStocker incorpora un módulo de predicciones como una de sus funcionalidades principales. El sistema aplica técnicas de aprendizaje automático que permiten transformar datos históricos y registros actuales en estimaciones de ventas futuras. 

El objetivo de esta funcionalidad es reemplazar las decisiones basadas únicamente en la experiencia o la intuición por un enfoque sistemático apoyado en datos, incrementando la objetividad en la gestión del inventario. Además, su integración con fuentes externas de información le otorga la capacidad de adaptarse a contextos cambiantes, como la estacionalidad, los feriados o incluso factores climáticos que pueden influir en el consumo.

\subsection{Fuentes de Datos}\label{sec:fuentes-datos}

El modelo predictivo requiere datos de calidad para generar resultados confiables. Con este fin, la plataforma web admite dos vías principales de recopilación:

\begin{itemize}
    \item \textbf{Carga manual de datos mediante archivos CSV}: esta modalidad permite a los locales gastronómicos incorporar sus registros históricos directamente desde sistemas internos o planillas de control. Para este fin, se determinó que el formato esperado para archivos CSV es:
    \begin{itemize}
        \item Fecha venta: dd/mm/aaaa hh:mm:ss
        \item Producto: con el mismo nombre usado al generar el producto en la aplicación.
        \item Cantidad.
    \end{itemize}

    \item \textbf{Integración automática con sistemas de venta}: este mecanismo posibilita la importación continua de pedidos en tiempo real desde aplicaciones como PedidosYa. Con ello, el sistema asegura la actualización permanente del conjunto de datos, reduciendo la dependencia exclusiva de registros históricos y adaptándose a la dinámica diaria del negocio.
\end{itemize}

\subsection{Gestion de ingredientes e items del menu}\label{sec:carga-datos}

Para que el sistema pueda calcular el inventario sugerido, es necesario que el usuario cargue los ingredientes que utiliza en su local y los vincule con los productos gastronómicos que ofrece en su menú. Esta funcionalidad permite al usuario definir y administrar tanto los ingredientes como los items del menu, permitiendo, en el caso particular de los ingredientes, realizar esta carga no solo desde la UI, sino tambien de forma masiva a traves de un archivo, a fin de simplificar la operatoria en caso de que el usuario cuente con un listado extenso.

\subsection{Procesamiento Predictivo}\label{sec:procesamiento-predictivo}

El procesamiento predictivo constituye el núcleo técnico de la funcionalidad de SmartStocker. Una vez recopilados los datos históricos y actuales, el sistema los somete a un proceso de depuración y normalización que asegura la consistencia y homogeneidad de los registros. Este preprocesamiento incluye la eliminación de valores atípicos, la imputación de datos faltantes y la transformación de variables temporales (como la estacionalidad o los turnos horarios) en características relevantes para el modelo.

Posteriormente, los datos son utilizados para entrenar algoritmos de aprendizaje supervisado, dentro de los cuales se evaluaron diferentes enfoques, priorizando modelos de regresión y técnicas de boosting como CatBoost, debido a su robustez en contextos con variables categóricas y a su capacidad de reducir el overfitting. Estos modelos generan proyecciones de ventas futuras en base a patrones identificados, incorporando tanto las tendencias históricas como variables externas tales como clima, feriados y días de la semana.

Las predicciones se ejecutan bajo demanda, es decir, cada vez que el usuario lo requiera dentro de la plataforma, lo que garantiza resultados actualizados y adaptados al contexto puntual de planificación del negocio, integrando un mecanismo de retroalimentación continua (feedback loop) mediante el cual los usuarios validan si las estimaciones reflejaron la demanda real observada; esa información se reintegra al modelo para ajustar progresivamente su precisión, personalizando las predicciones al contexto específico de cada restaurante y favoreciendo la evolución de SmartStocker ante escenarios cambiantes.

En consecuencia, el procesamiento predictivo de SmartStocker no se limita a entregar un valor numérico de ventas estimadas, sino que constituye un sistema adaptable y evolutivo, orientado a respaldar la toma de decisiones estratégicas en la gestión de inventarios gastronómicos.

\subsection{Cálculo de Inventario Sugerido}\label{sec:calculo-inventario}

Una vez obtenidas las predicciones de ventas, los resultados obtenidos son vinculados con la lista de ingredientes definida por el restaurante. De esta manera, se genera un cálculo automático del inventario recomendado, minimizando tanto los riesgos de desabastecimiento como los costos derivados de un exceso de stock. 

\subsection{Notificación de Alertas}\label{sec:alertas}

El módulo de notificaciones cumple un rol preventivo. Cada vez que un insumo alcanza un nivel inferior al umbral configurado por el usuario, el sistema genera una alerta en el tablero principal y, en futuras versiones, podrá enviarlas vía correo electrónico o notificaciones push.

El valor de este componente radica en su capacidad para evitar interrupciones operativas. En un negocio gastronómico, la ausencia de un ingrediente clave no solo afecta la venta de un plato específico, sino que también impacta en la experiencia del cliente y en la reputación del local. Por esta razón, las alertas de stock bajo no deben considerarse únicamente como un aviso técnico, sino como un mecanismo de aseguramiento de la calidad del servicio.

\section{Tecnologías utilizadas}\label{sec:tecnologias-utilizadas}

SmartStocker se basa en una serie de tecnologías y herramientas que permiten su funcionamiento eficiente y escalable. A continuación, se detallan las principales tecnologías utilizadas en el desarrollo de la solución:

\subsection{Next.js}\label{sec:nextjs}

Framework basado en React que permite crear aplicaciones web optimizadas con renderizado del lado del servidor. El uso de React se basa en que es ampliamente utilizado en el ecosistema de desarrollo frontend actual, lo que simplifica su mantenimiento a futuro y nos provee de múltiples librerías para utilizar en el desarrollo.

\subsection{Express}\label{sec:express}

Framework de Node.js que facilita la creación de aplicaciones web y APIs. Utilizado debido a que es fácilmente desplegable de forma serverless, y porque no se detectó la necesidad de utilizar un lenguaje fuertemente tipado.

\subsection{Python}\label{sec:python}

Lenguaje de programación de alto nivel, versátil y ampliamente utilizado en ciencia de datos y machine learning. Su uso radica en la enorme cantidad de librerías disponibles para realizar trabajos con datasets, tales como Panda y Numpy, entre otras.

\subsection{Amazon Web Services}\label{sec:aws}

Para el despliegue y operación de SmartStocker se utiliza Amazon Web Services (AWS), una plataforma en la nube que permite implementar una arquitectura escalable, segura y altamente disponible. Dentro del ecosistema de AWS se emplean los siguientes servicios principales

\begin{itemize}
    \item \textbf{AWS Amplify:} Plataforma que simplifica la creación, configuración y despliegue de aplicaciones web y móviles, integrando backend, hosting y flujos de desarrollo continuo. Se optó por su uso a fin de simplificar lo más posible los aspectos relacionados al despliegue del frontend, su capacidad de escalamiento ante la demanda, y su soporte nativo para Next.js.

    \item \textbf{Amazon S3 (Simple Storage Service):} Servicio de almacenamiento de objetos que permite guardar y recuperar grandes volúmenes de datos de forma segura, duradera y altamente disponible. Se optó por su uso debido a que, por su bajo costo y durabilidad, nos permite guardar de forma segura los datos históricos y modelos entrenados para cada usuario.

    \item \textbf{AWS Lambda:} Servicio de computación serverless que permite ejecutar código en respuesta a eventos sin necesidad de administrar servidores. Se decidió su uso a fin de facilitar la gestión de los procesos que corren en los pipelines y para reducir costos, dado que AWS factura su uso únicamente en base a las ejecuciones realizadas.

    \item \textbf{Amazon EventBridge:} Servicio de gestión de eventos que facilita la conexión entre aplicaciones y la automatización de flujos mediante reglas y programación de tareas. Cumple de forma nativa con nuestro requerimiento de realizar entrenamientos de forma periódica.

    \item \textbf{Amazon SageMaker:} Plataforma integral de machine learning que permite construir, entrenar y desplegar modelos de Machine Learning. Como la solución nativa de AWS para este tipo de aplicaciones, nos provee la capacidad de ejecutar nuestros procesos dentro de su entorno, lo que permite contar con una mayor disponibilidad de recursos en comparación con otras alternativas, como EC2 o Lambda.

    \item \textbf{Amazon DocumentDB:} Servicio de base de datos NoSQL compatible con MongoDB, diseñado para almacenar, consultar y escalar datos en formato de documentos JSON. Ofrece alta disponibilidad, seguridad integrada y capacidad de escalado automático. Se decidió optar por su uso tanto por la flexibilidad que otorga en términos de esquema como por su integración natural con Node.js.

    \item \textbf{Amazon SQS (Simple Queue Service):} Servicio de colas de mensajería que permite desacoplar y coordinar componentes de sistemas distribuidos. Facilita la comunicación asíncrona, el procesamiento escalable y la tolerancia a fallos entre servicios. Se incluyó en la solución a fin de ejecutar ciertos procesos de forma asíncrona, separados del componente que los inició originalmente, como el procesamiento de una venta.
\end{itemize}

\section{Arquitectura de la solución}\label{sec:arquitectura-solucion}
En esta sección se presenta la arquitectura de SmartStocker, detallando como las tecnologías empleadas se implementan en el diseño técnico y funcional. La arquitectura funciona como marco conceptual que sustenta cada componente del sistema y especifica las interacciones y dependencias entre las distintas partes que lo componen.

\subsection{Diagrama de Arquitectura conceptual}\label{sec:arquitectura-conceptual}
En relación con la arquitectura de alto nivel de SmartStocker, se propone un modelo de tres capas que organiza la solución, asegurando una separación clara de responsabilidades y mejorando la escalabilidad. Esta estructura está diseñada para optimizar tanto la interacción con el usuario como el procesamiento y almacenamiento de los registros de venta y los resultados de inferencia generados por el modelo predictivo.

La primera capa —Capa de Presentación (Presentation Layer)— agrupa la interfaz y la experiencia de usuario. En ella se implementa el frontend de SmartStocker con Next.js y se despliega mediante AWS Amplify, aprovechando sus capacidades de hosting y CI/CD. Esta capa permite a los usuarios interactuar con la plataforma, gestionar sus productos e ingredientes, visualizar recomendaciones y alertas, y realizar operaciones de predicciones de venta de forma responsiva y orientada a la operativa diaria.

La Capa de Negocio (Business Layer) constituye el núcleo lógico de SmartStocker y se organiza en tres componentes principales. El primero es una API REST que actúa como puente entre la capa de presentación y los servicios backend; a través de ella se gestionan las sesiones y la autorización, se exponen los endpoints para solicitar predicciones y gestionar productos, ingredientes, y alertas, y se incluye la lógica necesaria para atender las demandas de la interfaz de usuario de forma segura y consistente. Esta se encuentra desarollada con Node.js, a fin de aprovechar lo mas posible su integracion nativa con Next.js, y la simpleza a la hora de desplegar la API de forma serverless mediante AWS Lambda. El segundo es el pipeline de ETL, encargado de recibir los datos de ventas en tiempo real desde los sistemas externos, procesarlos, enriquecerlos, y dejarlos listos para ser usados por el modelo. Por último, el pipeline de ML será el encargado de realizar el entrenamiento del modelo, ya sea de forma periódica u bajo demanda, y de dejarlo disponible para su uso en las operaciones de inferencia. La logica de ambos pipelines se encuentra desarollada en Python, a fin de aprovechar las distintas librerías y frameworks disponibles para el procesamiento de datos y machine learning, y se encuentran implementados mediante AWS Lambda.

Por último, la Capa de Almacenamiento (Storage Layer) se organiza según los distintos casos de uso de la plataforma y está diseñada para optimizar rendimiento y costos. Los datos de la aplicación, tales como los ingredientes, productos, ventas realizadas,  y resultados de predicciones, se almacenan en DocumentDB, lo que facilita lecturas de baja latencia y flexibilidad en la estructura de los datos. Mientras tanto, los datos orientados al entrenamiento del modelo, tales como los datasets históricos, los artefactos de entrenamiento y los modelos exportados se almacenan en Amazon S3, aprovechando su durabilidad y su bajo costo.

\begin{figure}[H]
    \centering
    \includegraphics[width=1\textwidth]{images/arquitectura_capas.png}
    \caption{Arquitectura conceptual de SmartStocker.}
    {\textit{Fuente: Elaboración propia.}}
    \label{fig:arquitectura-conceptual}
\end{figure}

\subsection{Diagrama de Despliegue}\label{sec:arquitectura-despliegue}
Se decidió usar Amazon Web Services (AWS) como solución de despliegue debido a la flexibilidad y amplitud de herramientas ofrecidas necesarias para implementar la arquitectura de SmartStocker.

Analizando la capa de presentación, esta se encuentra desarrollada con Next.js, y desplegado a través de AWS Amplify. Se eligió este servicio puesto que simplifica enormemente la gestión de la aplicación, permitiendo, mediante una simple configuración inicial, encargarse de aspectos tales como el despliegue (permitiendo CI/CD integrado a GitHub), hasta del escalamiento en sí.

Para la capa de almacenamiento, se optaron por dos soluciones. Primero, DocumentDB como base de datos NoSQL, seleccionado debido a la naturaleza no estructurada de los datos a utilizar, y que nos brinda la posibilidad de modificar el schema con facilidad, además de ser una base serverless escalable. Y segundo, S3, para contener la información en formato \verb|csv| requerida para los entrenamientos del modelo, y para almacenar los archivos correspondientes al modelo entrenado en si.

Para la capa de negocio, se decidió implementar las distintas APIs requeridas mediante AWS Lambda, dado que su enfoque serverless permite abstraernos de la gestión de la infraestructura, acelerando el desarrollo y la puesta en producción, escalando cuando la demanda lo requiera, y reduciendo los costos fijos.

Esto también aplica para el pipeline de ETL, donde también se utiliza AWS Simple Queue Service (SQS) para desacoplar el procesamiento de las ventas, a fin de lograr velocidades de respuesta rápida ante los sistemas externos que enviaran las mismas, mientras que el procesamiento y enriquecimiento de las ventas se realiza en otra Lambda.

Por último, se decidió usar AWS Sagemaker en el pipeline de ML, puesto que simplifica el entrenamiento del modelo, requiriendo suministrarle los dataset de entrenamiento y validación a utilizar, y permite disponibilizar el modelo para ser usado para predicciones mediante endpoints, lo que brinda una consulta rápida y de facil implementacion.

\begin{figure}[H]
    \centering
    \includegraphics[width=1\textwidth]{images/arquitectura_despliegue.png}
    \caption{Arquitectura de Despliegue y Procesos de SmartStocker.}
    {\textit{Fuente: Elaboración propia.}}
    \label{fig:arquitectura-despliegue}
\end{figure}

\subsection{Diagrama de Base de datos}\label{sec:arquitectura-base-datos}

Para el diseño de la base de datos, se opto por utilizar DocumentDB, una de las alternativas NoSQL brindadas por AWS, debido a la flexibilidad y capacidades de escalado automatico que esta nos provee, siendo la flexibilidad del schema algo critico dada la necesidad de agregar campos adicionales conforme el proceso de ETL o entrenamiento del modelo de ML lo requieran.

El diseño de la base de datos se estructura en torno a siete entidades principales: Usuario, ItemMenu, Ingrediente, ItemMenu\_Ingrediente, Venta, DetalleVenta,CorreccionPrediccion y Alerta.
La entidad Usuario representa a cada negocio dentro de la plataforma, almacenando información básica de acceso (nombre de usuario, email, contraseña), su estado de actividad y los identificadores de integración con plataformas externas como PedidosYa, Rappi u OláClick.

La entidad Ingrediente contiene los insumos del negocio, registrando su nombre, unidad de medida, cantidad en stock, cantidad mínima y unidad de stock, además del userId propietario, mientras que ItemMenu registra los productos ofrecidos (código, nombre, estado activo) y contiene la lista de ingredientes necesarios para cada receta. Esa relación $N..M$ entre ItemMenu e Ingrediente se materializa en la tabla/intersección (el subdocumento de ingredientes) que guarda, por cada par producto-ingrediente, la cantidad\_requerida. 

La entidad Predicción registra las predicciones de demanda vinculadas a un producto\_id (ItemMenu) con fecha\_prediccion, turno, cantidad\_predicha y el userId que la generó.

La entidad Venta registra las transacciones realizadas, incluyendo el número de venta, fecha, turno, método de pago, plataforma de venta, estado y cantidad total de ítems, todo asociado a un userId y a su vez una se descompone en varios ítems dentro de la entidad DetalleVenta, donde se detallan el producto (ItemMenu), la cantidad, el precio unitario y el subtotal, lo que permite realizar análisis más precisos por producto y turno.

La entidad CorreccionPrediccion guarda el feedback explícito del usuario sobre los valores predichos por el modelo. Cada registro asocia un itemMenuId y un userId con la predicción original, el valor corregido, la fecha y el turno, permitiendo incluir el feedback del usuario en la prediccion.

Finalmente, la entidad Alerta gestiona las alertas de stock, asociando cada una a su correspondiente a usuarios e ingredientes, con indicadores de lectura y fechas de creación y actualización.
Todos los documentos incluyen timestamps (fecha\_creacion / fecha\_actualizacion) para auditoría.

\begin{figure}[H]
    \centering
    \includegraphics[width=1\textwidth]{images/arquitectura-base-datos.png}
    \caption{Arquitectura de Base de Datos de SmartStocker.}
    {\textit{Fuente: Elaboración propia.}}
    \label{fig:arquitectura-base-datos}
\end{figure}

\subsection{Pipeline ETL}\label{sec:pipeline-etl}

Con el objetivo de permitir la ingesta automática de las ventas, a fin de actualizar tanto los dashboards visualizados en la aplicación, activar las correspondientes alertas de stock si fuera necesario y disponibilizar la información para el entrenamiento del modelo predictivo, se implementó un pipeline de ETL mediante el cual se procesan las distintas ventas. La primera parte de este proceso es recibir y procesar las ventas que ocurren, para lo cual se diseñaron distintas Lambdas, ajustadas al distinto formato en el que las plataformas pueden enviar los datos de una venta realizada. En estas se analiza la información recibida y en base a ello se genera una nueva entrada en la entidad Ventas. Dado que estas instancias Lambda estarán atendiendo peticiones de sistemas externos, se decidió que el procesamiento que ocurre en ellas sea lo más rápido y sencillo posible, a fin de devolver una respuesta a la brevedad.

En base a esto, se implementó el uso de AWS SQS, permitiendo de esta forma que el resto de procesamiento requerido para transformar la venta en el dato requerido por el modelo ocurra en una Lambda distinta, que funciona como consumidor de la cola implementada en SQS, donde al finalizar la primer Lambda se encola un mensaje conteniendo el Id de la venta creada en DocumentDb, para que esta sea procesada por el consumidor. Esto nos brinda un procesamiento asincrónico y desacoplado de estas tareas que pueden ser más lentas, y permite que ocurran a un ritmo distinto de la Lambda inicial expuesta al resto de los sistemas de venta.

Esta segunda Lambda se encargará de procesar la venta, actualizando los niveles de stock y generando alertas de nivel del mismo si fuera necesario, para cada ingrediente utilizado en cada uno de los productos que integran cada venta.

Por último, al finalizar esta Lambda la información de cada una de las ventas es almacenada en S3 en formato CSV, generando una fila por cada producto involucrado en cada una de las ventas, disponible para ser usada al momento de entrenar el modelo, durante el paso de consolidación histórica, donde la información correspondiente a cada usuario se encontrará separada mediante el identificador único del usuario, a fin de garantizar la integridad de la información en los usos subsecuentes.

\subsection{Pipeline Machine Learning}\label{sec:pipeline-machine-learning}

El modelo predictivo de Machine Learning se encuentra diseñado de manera tal de poder considerar tres factores claves:

\begin{itemize}
    \item Factores cronológicos, siendo estos el día, semana y mes de la venta, como también el turno de trabajo en el cual fue realizada, y si la fecha corresponde a un feriado o no.
    \item Factores climáticos, específicamente temperatura promedio, la ocurrencia o no de lluvia, y la estación del año.
    \item La tendencia de cada producto, considerando como han sido las ventas de cada producto tanto de forma independiente, como también considerando la combinación producto y turno, siendo esto evaluado en el plazo de una semana y de un mes.
\end{itemize}

A fin de obtener la información relacionada a feriados y aspectos climáticos, se usan las APIs libres de ArgentinaDatos y Open-Meteo, respectivamente.

El algoritmo utilizado para el entrenamiento es CatBoost, perteneciente a la familia de algoritmos de regresión, seleccionado debido a su manejo nativo de variables categóricas.

Para la ejecución del entrenamiento en sí, se utilizan SageMaker Jobs, específicamente Processing Jobs para las tareas de Feature Engineering, y Training Jobs para el Entrenamiento propiamente del modelo.

Para ser capaces de realizar todo el proceso de entrenamiento de forma individual para cada usuario, se optó por implementar un pipeline, cuyo objetivo es entrenar de forma periódica un modelo actualizado para cada uno de ellos.

El mismo se encuentra implementado mediante el uso de AWS EventBridge, a fin de programar entrenamientos periódicos de forma semanal. Este disparará un evento que activará la primera Lambda, dedicada a consolidar los datos de la semana, y unificarlos con el dataset histórico ya existente. Como parte de esta, también se obtendrán los datos correspondientes a feriados y aspectos climáticos.

La segunda Lambda es la dedicada propiamente a coordinar las distintas fases del proceso de entrenamiento, Feature Engineering y Entrenamiento, cuya lógica se encuentra definida en sus respectivos scripts en Python.

Específicamente, sus funciones son las siguientes:

\begin{itemize}
    \item Feature engineering: Aquí se tomará el archivo histórico consolidado del usuario, y se obtendrán features adicionales para enriquecer el modelo, siendo estas las relacionadas a aspectos cronológicos y de tendencias de productos. También se realizan tratamiento de datos anómalos, y normalizaciones. La salida de este proceso son los dos datasets a utilizar para el entrenamiento del modelo, uno para el entrenamiento propiamente, y otro para su correspondiente validación, depositados dentro de S3.
    \item Entrenamiento: Encargado de utilizar los dos datasets creados previamente, e instanciar y ejecutar el entrenamiento en sí. Su salida es el archivo correspondiente al modelo ya entrenado.
\end{itemize}

La ventaja de que estas fases sean ejecutadas como SageMaker Jobs es que su performance aumenta notablemente en comparación de una ejecución en una Lambda estándar, puesto que este entorno cuenta con recursos ampliamente superiores en relación a ella.

Luego de estas etapas, se registra el modelo, asociando a él un tercer script de Python, que contiene la lógica a ejecutar al querer usarlo en una predicción, incluyendo el mínimo Feature Engineering requerido para que los datos a utilizar sean compatibles con el modelo. Y a su vez, este modelo queda asociado a un SageMaker Endpoint, que permite que la funcionalidad de predicción pueda ser realizada mediante peticiones HTTP a un endpoint dedicado.

\begin{figure}[H]
    \centering
    \includegraphics[width=1\textwidth]{images/diagramaPipelineML.png}
    \caption{Diagrama del Pipeline de Machine Learning.}
    {\textit{Fuente: Elaboración propia.}}
    \label{fig:diagrama-pipeline-ml}
\end{figure}

\subsection{Feature Engineering}\label{sec:feature-engineering}

Dentro del proceso de Feature Engineering, se generan una serie de features adicionales a partir de los datos históricos de ventas, con el objetivo de enriquecer el modelo y permitirle capturar patrones relevantes en la demanda, partiendo de los siguientes datos correspondientes al dataset generado en base a las ventas:

\begin{table}[H]
\centering
\caption{Datos base del dataset de ventas.}
\label{tab:dataset-base}

\renewcommand{\arraystretch}{1.25}
\begin{tabular}{|l|l|}
\hline
\textbf{Campo} & \textbf{Descripción} \\ \hline
cantidad & Cantidad vendida de un producto para una determinada fecha. \\
 & Variable objetivo a predecir por el modelo \\ \hline
fecha & Fecha de la venta en formato DD/MM/YYYY \\
 & (usado para agrupación y features temporales) \\ \hline
producto & Código del producto/ítem del menú \\ \hline
turno & Turno de venta \\ \hline
\end{tabular}

\vspace{0.3em}
\captionsetup{justification=centering}
{\small \textit{Fuente: Elaboración propia.}}
\end{table}

A partir de estos datos, se generan los siguientes tipos de features adicionales:

\textbf{Features Temporales:} Representan los aspectos cronológicos de una venta. A fin de capturar la naturaleza cíclica de estos datos, se optó por representar a las variables de día, semana y mes mediante el resultado de aplicar las funciones trigonométricas seno y coseno a sus correspondientes valores numéricos, de esta forma, se logra que el modelo pueda captar la continuidad entre el final y el comienzo de cada ciclo (por ejemplo, entre el día 31 y el día 1).

\textbf{Features Climáticas:} Representan los aspectos climáticos del momento donde se concretó una venta.

\textbf{Features de Lag:} Corresponden a variables que incorporan información de ventas pasadas en un intervalo temporal específico, permitiendo que el modelo aprenda patrones de comportamiento recurrentes en el tiempo. Por ejemplo, la variable \textit{lag\_1\_semana} representa el valor de ventas observado exactamente una semana antes del registro actual.

\textbf{Rolling Features:} Estas variables se obtienen a partir del cálculo de estadísticas móviles (como medias o desviaciones estándar) sobre ventanas temporales deslizantes. Su objetivo es reflejar la tendencia y la estabilidad reciente de las ventas, suavizando las fluctuaciones diarias. De esta forma, el modelo puede identificar cambios graduales en la demanda y diferenciar entre comportamientos estables y volátiles a corto o mediano plazo. En este caso, se calcularon medias(mean) y desviaciones estándar móviles(std) para ventanas de 1 semana(1w), 2 semanas(2w) y 4 semanas(4w), para cada producto(p), y las combinaciones producto-turno(pt) y producto-día(pdw).

Al finalizar el proceso de Feature Engineering, el dataset resultante contiene las siguientes variables:

\begin{table}[H]
\centering
\caption{Features Temporales.}
\label{tab:features-temporales}

\renewcommand{\arraystretch}{1.25}
\begin{tabular}{|l|l|}
\hline
\textbf{Feature} & \textbf{Descripción} \\ \hline
semana & Semana del mes \\ \hline
semana\_cos & Coseno del valor numérico de la semana \\ \hline
semana\_sin & Seno del valor numérico de la semana \\ \hline
mes & Número del mes (1-12) \\ \hline
mes\_cos & Coseno del valor numérico del mes \\ \hline
mes\_sin & Seno del valor numérico del mes \\ \hline
dia\_mes & Día del mes (1-31) \\ \hline
dia\_mes\_cos & Coseno del valor numérico del día del mes \\ \hline
dia\_mes\_sin & Seno del valor numérico del día del mes \\ \hline
es\_fin\_de\_semana & Indicador si es sábado o domingo \\ \hline
es\_feriado & Indicador si es feriado argentino \\ \hline
estacion & Estación del año en Hemisferio Sur (categórica) \\ \hline
\end{tabular}

\vspace{0.3em}
\captionsetup{justification=centering}
{\small \textit{Fuente: Elaboración propia.}}
\end{table}

\begin{table}[H]
\centering
\caption{Features Climáticas.}
\label{tab:features-climaticas}

\renewcommand{\arraystretch}{1.25}
\begin{tabular}{|l|l|}
\hline
\textbf{Feature} & \textbf{Descripción} \\ \hline
temperatura\_promedio & Categoría de temperatura \\ \hline
llovio & Indicador si llovió en el día \\ \hline
clima\_adverso & Indicador de condiciones climáticas adversas \\ \hline
descripcion\_clima & Descripción textual del clima \\ \hline
\end{tabular}

\vspace{0.3em}
\captionsetup{justification=centering}
{\small \textit{Fuente: Elaboración propia.}}
\end{table}

\begin{table}[H]
\centering
\caption{Features de Lag.}
\label{tab:features-lag}

\renewcommand{\arraystretch}{1.25}
\begin{tabular}{|l|l|}
\hline
\textbf{Feature} & \textbf{Descripción} \\ \hline
lag\_1\_semana & Total de ventas para la combinación de producto y turno \\ 
& exactamente una semana atrás \\ \hline
lag\_4\_semanas & Total de ventas para la combinación de producto y turno \\ 
& exactamente cuatro semanas atrás \\ \hline
\end{tabular}

\vspace{0.3em}
\captionsetup{justification=centering}
{\small \textit{Fuente: Elaboración propia.}}
\end{table}

\begin{table}[H]
\centering
\caption{Rolling Features.}
\label{tab:features-rolling}

\renewcommand{\arraystretch}{1.25}
\begin{tabular}{|l|l|}
\hline
\textbf{Feature} & \textbf{Descripción} \\ \hline
roll\_mean\_[ventana]\_[combinacion] & Promedio móvil de ventas \\ 
 & para ventanas de 1w, 2w y 4w \\ 
 & con combinaciones producto(p), producto-turno(pt) \\
 & y producto-día(pdw) \\ \hline
roll\_std\_[ventana]\_[combinacion] & Desviación estándar móvil de ventas \\ 
 & para ventanas de 1w, 2w y 4w \\
 & con combinaciones producto(p), producto-turno(pt) \\
 & y producto-día(pdw) \\ \hline
\end{tabular}

\vspace{0.3em}
\captionsetup{justification=centering}
{\small \textit{Fuente: Elaboración propia.}}
\end{table}

\subsection{Entrenamiento del modelo}\label{sec:entrenamiento-modelo}

Como parte del entrenamiento, se decidio usar la funcion de perdida Tweedie, debido a su capacidad para modelar variables continuas no negativas con distribución asimétrica y presencia de ceros, como ocurre en las ventas diarias de productos gastronómicos, donde es una ocurrencia normal que ciertos productos no sean vendidos en determinados días. Esto permite que el modelo se adapte mejor a estos patrones, mejorando la precisión de las predicciones.

A continuacion, se muestra un grafico donde se visualiza la importancia de las diez features mas relevantes, calculado usando la funcion PredictionValuesChange de CatBoost, la cual mide el impacto de cada feature en las predicciones del modelo al observar los cambios en los valores predichos cuando se modifica cada feature individualmente.

\begin{figure}[H]
    \centering
    \includegraphics[width=1\textwidth]{images/importancia.png}
    \caption{Importancia de las features en el modelo.}
    {\textit{Fuente: Elaboración propia.}}
    \label{fig:importancia-features}
\end{figure}

El modelo se entrenó con \textbf{2000 iteraciones} y una \textbf{tasa de aprendizaje de 0.0347}, lo que permite una convergencia progresiva sin overfitting. La \textbf{profundidad de los árboles (5 niveles)} limita la complejidad y mejora la estabilidad del modelo. Se utilizó una \textbf{regularización L2 de 44.95} y una \textbf{temperatura de muestreo de 1.79}, favoreciendo la diversidad de los árboles y reduciendo la varianza.  

El entrenamiento incorpora la técnica de \textit{early stopping} con \textbf{100 iteraciones}, de manera tal que el entrenamiento es finalizando cuando no se observan mejoras significativas entre iteraciones. Se empleó la política de crecimiento \textit{SymmetricTree}, propia de CatBoost, que genera árboles balanceados y eficientes.

Usando estos hiperparametros, logramos los siguientes resultados al evaluar el modelo:

\begin{figure}[H]
    \centering
    \includegraphics[width=1\textwidth]{images/resultados.png}
    \caption{Resultados del modelo.}
    {\textit{Fuente: Elaboración propia.}}
    \label{fig:resultados-modelo}
\end{figure}

Podemos ver que tanto el RMSE como el MAE presentan valores bajos, lo que indica que el modelo es apto para la tarea de predicción de ventas enfocada a la gestion de inventario.

\subsection{Inclusión del Feedback de usuario}\label{sec:inclusion-feedback-usuario}

A fin de incluir el feedback del usuario en las predicciones, se decidió implementar una lógica que ajuste la salida del modelo. Para ello, cada vez que el usuario informe que una predicción no es precisa en base a su experiencia, podrá informar el valor de ventas que él considera adecuado, ajustando esto el cálculo de stock requerido. Cuando esto ocurra de forma reiterada para una cierta combinación producto/turno dentro de un plazo de tiempo, se utilizaran los valores corregidos para calcular un valor de ajuste, teniendo más peso las correcciones recientes, y se aplicará a la predicción obtenida del modelo. 

Esto permite realizar ajustes sobre posibles tendencias que están ocurriendo de forma más rápida de la que el modelo es capaz de captarlas, sin modificar el proceso de entrenamiento del mismo, ya que eventualmente, si estos cambios en las tendencias pasarán a ser el comportamiento normal, el modelo lo aprenderá de forma natural conforme va accediendo a nueva información.

\section{Metodología de Desarrollo}\label{sec:metodologia-desarrollo}

El desarrollo de SmartStocker se realizó bajo el modelo de desarrollo en cascada, una metodología tradicional que organiza el trabajo en etapas secuenciales y claramente definidas. Cada fase depende del cumplimiento de la anterior, lo que permite mantener un flujo ordenado, documentado y con una trazabilidad precisa de los avances. Se consideró que este enfoque era el adecuado, dado que, luego del User Research, se cuenta con requerimientos bien definidos, lo que favoreció la planificación y la validación progresiva de resultados.

\subsection{Análisis y definición de requerimientos}\label{sec:analisis-definicion-requerimientos}

Tomando como base el User Research realizado, se comenzó con la identificación detallada de las necesidades del sistema, tanto funcionales como técnicas. Se establecieron los objetivos generales (optimización del stock y predicción de ventas en el sector gastronómico) junto con los requerimientos específicos de usabilidad, almacenamiento y procesamiento de datos. Esta etapa permitió definir con claridad el alcance del sistema y reducir incertidumbres en fases posteriores.

\subsection{Diseño conceptual y arquitectónico}\label{sec:diseño-conceptual-arquitectonico}

En esta instancia se elaboró la arquitectura general de SmartStocker, estructurada en tres capas: presentación, negocio y almacenamiento. Se seleccionaron tecnologías compatibles entre sí dentro del ecosistema AWS, priorizando la escalabilidad, la modularidad y la integración entre servicios. El diseño incluyó diagramas de componentes, flujos de datos y la definición de las interacciones entre los distintos módulos. Esto también incluyó las definiciones de qué aspectos serían considerados por el modelo de Machine Learning.

\subsection{Implementación y desarrollo}\label{sec:implementacion-desarrollo}

Durante esta fase se construyeron los distintos componentes del sistema siguiendo las especificaciones del diseño. El frontend fue desarrollado en Next.js y gestionado mediante AWS Amplify, mientras que la lógica de negocio se implementó en Node.js sobre funciones Lambda. Los modelos de machine learning se entrenaron y desplegaron utilizando Amazon SageMaker, integrando los flujos de datos procesados y almacenados en DocumentDB y S3.

\subsection{Validación e integración final}\label{sec:validacion-integracion-final}

Finalizada la implementación, se llevaron a cabo pruebas de validación funcional, de rendimiento y de integración entre servicios. Se verificó el correcto funcionamiento de las integraciones, la coherencia de las predicciones y la estabilidad del entorno de despliegue, junto con el cumplimiento de los requerimientos tanto funcionales como no funcionales.

	\chapter{Conclusi\'on}\label{conclussions}

\section{Resumen de aportes}

Este trabajo ha abordado la problemática de la gestión ineficiente del inventario en el sector gastronómico de la Ciudad Autónoma de Buenos Aires, poniendo especial atención en las limitaciones que enfrentan los pequeños y medianos establecimientos al no disponer de herramientas accesibles que les permitan anticipar la demanda y optimizar el uso de sus recursos. A través de un proceso integral de investigación teórica, análisis del estado del arte, trabajo de campo con actores del rubro y desarrollo técnico de una solución funcional, se ha confirmado la existencia de una necesidad concreta de incorporar tecnologías predictivas en la gestión de inventarios gastronómicos, con el fin de mejorar la rentabilidad y la sostenibilidad operativa.

La propuesta desarrollada, denominada SmartStocker, se posiciona como una plataforma web innovadora que aplica técnicas de aprendizaje automático para predecir las ventas y calcular los niveles de stock óptimos requeridos en función de dichas predicciones. A diferencia de las herramientas disponibles en el mercado argentino, que se enfocan principalmente en la administración de ventas y la generación de reportes descriptivos, SmartStocker introduce un enfoque predictivo y adaptativo capaz de incorporar variables externas como el clima, los feriados y el comportamiento histórico del consumidor para ofrecer estimaciones más precisas. De esta manera, contribuye a reducir el desperdicio de alimentos, evitar faltantes de insumos y sostener la disponibilidad del menú, factores que impactan directamente en la experiencia del cliente y en la competitividad del negocio.

A lo largo del proceso se abordaron distintas etapas metodológicas: desde la investigación del contexto del sector gastronómico y la recolección de datos empíricos mediante entrevistas y encuestas, hasta el diseño, desarrollo, validación e implementación de la solución. Las entrevistas con dueños de pymes gastronómicas evidenciaron las dificultades de controlar el inventario con métodos manuales, así como los sobrecostos asociados al error humano y la falta de integración entre plataformas. Por su parte, las encuestas a consumidores confirmaron el impacto directo que tiene la falta de stock en la satisfacción y fidelización del cliente. Estos hallazgos sirvieron como base empírica para orientar el diseño del sistema, alineando la solución con las necesidades reales del mercado.

El análisis económico desarrollado en el capítulo final demostró la viabilidad financiera del proyecto, con indicadores positivos en los tres escenarios planteados (optimista, neutral y pesimista). Los resultados del valor actual neto, la tasa interna de retorno y el período de recuperación de la inversión evidencian que la implementación de SmartStocker es rentable y sostenible a mediano plazo, tanto para sus desarrolladores como para los potenciales usuarios del sistema. Asimismo, se destaca su escalabilidad técnica y comercial, ya que la solución puede adaptarse fácilmente a diferentes tipos de negocios gastronómicos y expandirse hacia otras regiones del país.

En conclusión, el desarrollo de SmartStocker logró integrar de manera efectiva la teoría, la práctica y la innovación. El trabajo permitió comprobar que la aplicación de modelos de aprendizaje automático en el sector gastronómico no solo es técnicamente factible, sino también útil y económicamente viable. En líneas generales, este proyecto deja en evidencia el potencial de las tecnologías predictivas para mejorar la gestión de recursos en entornos reales, y marca un punto de partida para futuras implementaciones más completas. SmartStocker representa una contribución concreta al campo de la ingeniería aplicada, y refleja cómo una solución desarrollada desde un enfoque académico puede transformarse en una herramienta práctica, con impacto real en la eficiencia y la sostenibilidad de los negocios gastronómicos.

\section{Trabajo futuro}

Al tratarse de un producto mínimo viable, el desarrollo actual de SmartStocker constituye la base de una solución que puede continuar evolucionando en futuras iteraciones. Existen diversas oportunidades de mejora y expansión que permitirían consolidar la plataforma y ampliar su alcance dentro del sector gastronómico.

Una primera línea de trabajo se orienta a la ampliación geográfica del sistema. Si bien el proyecto se desarrolló tomando como referencia el contexto gastronómico de la Ciudad Autónoma de Buenos Aires, su arquitectura permite escalar fácilmente hacia otros municipios del Gran Buenos Aires. La incorporación de nuevas zonas implicaría ajustar ciertos parámetros del modelo predictivo, considerando variaciones locales en el comportamiento del consumo, la estacionalidad y la disponibilidad de insumos. Este proceso de expansión territorial no solo permitiría validar la robustez del sistema, sino también aumentar su impacto y nivel de adopción.

En paralelo, se proyecta el desarrollo de una aplicación móvil complementaria que permita acceder a las principales funciones de la plataforma desde dispositivos Android o iOS. Esta versión facilitaría el control del inventario, la consulta de métricas y la recepción de alertas en tiempo real, ofreciendo mayor flexibilidad operativa a los usuarios y favoreciendo la adopción por parte de pequeños comercios que no cuentan con equipos de escritorio en su lugar de trabajo.

Otra mejora posible se relaciona con la experiencia del usuario. A medida que la plataforma sea utilizada por más establecimientos, resultará necesario optimizar los flujos de navegación, incorporar filtros personalizados y ampliar los reportes visuales para simplificar la interpretación de los datos. Estas mejoras apuntan a mantener una interfaz intuitiva y adaptada a diferentes perfiles de usuario, desde administradores hasta encargados de cocina o compras.

También se plantea la posibilidad de fortalecer las funcionalidades actuales mediante la incorporación de módulos complementarios. Entre ellos, se considera la automatización parcial de las órdenes de compra, la detección de tendencias de consumo específicas por categoría de producto, o la generación de alertas ante variaciones anómalas en la demanda. Estas funciones permitirían mejorar la eficiencia general del sistema y brindar un soporte más completo para la toma de decisiones.

Por último, una línea de avance relevante consiste en optimizar el modelo predictivo utilizado. Con una mayor cantidad de datos históricos disponibles, se podrá evaluar la incorporación de nuevas variables contextuales, como la temperatura, eventos locales o promociones estacionales, que podrían contribuir a aumentar la precisión de las predicciones. De igual manera, se prevé implementar mecanismos de validación periódica del modelo para asegurar su desempeño en escenarios cambiantes.

En síntesis, las líneas de trabajo propuestas no buscan transformar la esencia del sistema, sino fortalecer su funcionalidad, ampliar su alcance territorial y mejorar su usabilidad. La evolución de SmartStocker dependerá en gran medida del aprendizaje obtenido a partir de su implementación real, lo que permitirá ajustar y perfeccionar el producto de manera continua, consolidándolo como una herramienta confiable, accesible y adaptable para el sector gastronómico de la región.


	\addcontentsline{toc}{chapter}{Bibliograf\'{\i}a}
	\printbibliography{}

	\appendix
	\renewcommand{\appendixname}{Anexo}
\renewcommand{\thechapter}{\Alph{chapter}}

% Comando para iniciar la sección de anexos en el índice
\newcommand{\initappendix}{%
  \addcontentsline{toc}{chapter}{Anexo}%
}

% Comando para cada anexo individual
\newcommand{\anexo}[1]{%
  \refstepcounter{chapter}%
  \chapter*{Anexo \thechapter: #1}%
  \addcontentsline{toc}{section}{Anexo \thechapter: #1}%
  \markboth{Anexo \thechapter: #1}{}%
}

\initappendix

\anexo{Cronograma de Actividades}

Durante el desarrollo del proyecto, se realizó un ajuste en la planificación inicial vinculada a las etapas de relevamiento y validación. Originalmente, se había previsto iniciar la recolección de información mediante encuestas estructuradas. Sin embargo, en función de las necesidades emergentes del proyecto y con el objetivo de alcanzar una comprensión más profunda del contexto real, así como detectar patrones y problemáticas concretas del rubro gastronómico, se optó por priorizar una instancia cualitativa basada en entrevistas semiestructuradas. 

Como parte de esta decisión, se amplió la cantidad de entrevistas relevadas y por ende la duración de las mismas, incluyendo la transcripción y el análisis de dichas representaciones del sector.

Como consecuencia, la encuesta fue reprogramada para una segunda instancia, a fin de que las preguntas formuladas reflejen con mayor precisión los escenarios identificados, todo esto con el fin de favorecer una toma de decisiones más informada a lo largo del desarrollo del proyecto.

\begin{figure}[ht]
    \centering
    \includegraphics[width=0.7\textwidth]{images/cronograma.png}
    \caption{Cronograma actualizado de actividades.}
    \label{fig:cronograma}
\end{figure}
 % En las entregas que no son la entrega final, se debe tener al menos un anexo con el cronograma (Gantt)
\anexo{Resultados de la encuesta}

Con el objetivo de comprender los hábitos de consumo en locales gastronómicos y la percepción de los clientes respecto a la disponibilidad de los productos, se realizó una encuesta a más de 150 personas. El propósito de la misma fue poder identificar cómo la falta de stock y la planificación de inventario influyen directamente en la satisfacción y fidelización de los consumidores, un factor clave en el mundo gastronómico.

En cuanto a la frecuencia de consumo, el 69.8\% de los encuestados afirmó visitar locales gastronómicos o pedir por delivery al menos una vez por semana, lo que refleja una relación constante con este tipo de establecimientos.

Respecto a la disponibilidad de productos, el 47.8\%, es decir casi la mitad de los entrevistados señaló que en el último mes le sucedió que el plato o producto deseado no se encontraba disponible. Ahora bien, la falta de stock recurrente sí reflejó tener un impacto notable en la percepción del cliente ya que el 86,8\% indicó que reduciría sus visitas si un local no mantiene la disponibilidad de los platos que ofrece. Además, mientras que el 30,2\% expresó que esta situación le genera desconfianza y afecta negativamente su experiencia, el 62,92\% afirmó que le molesta al menos un poco cuando un plato no está disponible. Si se consideran ambos grupos, puede observarse que más del 93\% de los encuestados experimenta algún grado de molestia o descontento frente a la falta de disponibilidad, lo que evidencia el peso crítico de este factor en la satisfacción del cliente.

Por otra parte, el rol de la planificación de compras resultó clave: el 98,7\% de los participantes coincidió en que una mejor gestión del inventario puede mejorar significativamente el servicio. En la misma línea, el 99,4\% declaró que estaría más dispuesto a regresar a un local que siempre mantenga su menú disponible y con calidad constante, y el 98,7\% lo recomendaría más a terceros.

Finalmente, la encuesta mostró que la falta frecuente de platos afecta directamente la reputación de los locales (71,7\% lo considera un factor muy relevante) y entre las respuestas abiertas, los clientes destacaron como principal fuente de satisfacción la combinación de calidad, disponibilidad, buena atención y relación precio-calidad, confirmando la importancia de un sistema que permita optimizar el control de insumos. % En el caso de tener en cuestas tiene que haber una transcripción de los resultados
% ============================
% Anexo: Entrevista 1 (Ulises)
% ============================
\anexo{Transcripción de la entrevista a Ulises Litterio (La Brava Burger)}

Ambas entrevistas fueron realizadas en Mayo de 2025.
% Contexto: Entrevista estructurada sobre gestión de inventario y demanda en gastronomía.
% Fecha de referencia declarada por el entrevistado: mayo 2025 (para valores de costos).
\begin{description}[leftmargin=0cm, labelsep=0.5cm]

  \item[\textbf{Entrevistador:}] ¿Podrías describir tu negocio, qué venden, cuántos locales y cuántos empleados?

  \item[\textbf{Ulises Litterio:}] El negocio es comercializar hamburguesas, con producción desde la compra de la materia prima, elaboración y venta con \textit{deliveries} propios. Actualmente tenemos 3 locales, con un total de 25 empleados entre todos. Contamos con varios proveedores para cada materia prima. Nuestros locales funcionan 6 días a la semana, en turnos de día y noche.

  \item[\textbf{Entrevistador:}] Puntualizando en inventario: ¿cómo lo controlan? ¿Usan software o planillas? ¿Cuándo realizan compras a proveedores?

  \item[\textbf{Ulises Litterio:}] El inventario se gestiona por local. Lo controla el jefe de cocina, quien informa al encargado, y éste lo carga en una planilla de Excel. Esa planilla la revisa el encargado de compras (el dueño). Las compras a proveedores se realizan una vez por semana, generalmente los lunes; si hay faltantes durante la semana, se repone. Los pagos son a semana vencida. Aún con organización, a veces no sabemos si efectivamente hay stock o si alcanzará para los pedidos; el jefe de cocina suele detectarlo en el momento y avisa en mostrador.

  \item[\textbf{Entrevistador:}] ¿Cómo calculan las cantidades o productos a comprar?

  \item[\textbf{Ulises Litterio:}] Nos basamos en las ventas de la semana anterior y en el momento del mes (a principio de mes suele haber más demanda). También consideramos feriados o eventos (por ejemplo, si llueve o hace mucho calor las ventas tienden a subir).

  \item[\textbf{Entrevistador:}] ¿Qué tan frecuente es el error en ese cálculo (por exceso o por defecto)?

  \item[\textbf{Ulises Litterio:}] Es común que falle. A veces sobreestockeamos para no quedarnos cortos (ya nos pasó), pero después sobra. Por ejemplo, en la semana del “día de la hamburguesa” estimamos 25 cajas de papas, pedimos 30 y terminamos usando 22: 8 de sobrestock, que compensamos la semana siguiente. Además, los proveedores manejan listas de precios por cantidad, así que comprar menos no siempre significa gastar menos (cambia el precio unitario o las condiciones).

  \item[\textbf{Entrevistador:}] ¿Cuál es el costo aproximado de reponer inventario por semana?

  \item[\textbf{Ulises Litterio:}] A la fecha, alrededor de 6 millones de pesos por cada local.

  \item[\textbf{Entrevistador:}] ¿Averiguaste algún software para gestionar?

  \item[\textbf{Ulises Litterio:}] Consulté por MaxiRest (incluye integraciones con PedidosYa y Mercado Pago), pero el costo era alto para una PyME y la información seguía quedando segmentada.

  \item[\textbf{Entrevistador:}] ¿Te serviría un software que, usando sus ventas, prediga cantidades a comprar y alerte por bajo stock?

  \item[\textbf{Ulises Litterio:}] Sí, sería muy útil para reducir el error de stock y avisar a tiempo cuando falte materia prima, sin depender tanto del control manual del jefe de cocina. La predicción debería considerar ventas e “instante del mes”. También sería clave integrar múltiples plataformas (usamos PedidosYa y Rappi), porque hoy los reportes están fragmentados y hay que consolidar todo manualmente.

\end{description}


% ==============================
% Anexo: Entrevista 2 (Oscar C.)
% ==============================
\anexo{Transcripción de la entrevista a Oscar Campione (Rotisería Familiar)}
% Contexto: Entrevista sobre operación diaria, compras y stock en rotisería familiar con delivery propio.
\begin{description}[leftmargin=0cm, labelsep=0.5cm]

  \item[\textbf{Entrevistador:}] ¿Podrías describir tu negocio, qué venden, cuántos locales y cuántos empleados?

  \item[\textbf{Oscar Campione:}] Es una rotisería familiar; vendemos productos hechos en el día. Trabajamos 100\% con delivery propio y PedidosYa. Tenemos un solo local (en parte de nuestra casa). El único empleado es el repartidor; el resto somos familiares (en total, 4 personas).

  \item[\textbf{Entrevistador:}] Sobre inventario: ¿cómo lo controlan? ¿Usan software o planillas? ¿Cada cuánto compran?

  \item[\textbf{Oscar Campione:}] Operamos día a día, así que las compras son diarias. Vendemos lo que hay y al día siguiente reponemos. Elegimos proveedores por mejor precio del momento y pagamos al contado (no compramos en cantidad). El inventario lo controlamos al terminar el turno noche, manualmente (producto por producto), y lo anotamos en un cuaderno para reponer por la mañana. No usamos soporte digital para inventario.

  \item[\textbf{Entrevistador:}] ¿Cómo calculan qué cantidades o productos comprar?

  \item[\textbf{Oscar Campione:}] Miramos lo vendido el día anterior (y a veces la semana anterior). Definimos un tope y tratamos de vender hasta ese límite; al día siguiente reponemos. Si alguien pide algo y no queda, tenemos que informarlo, lo que puede enojar al cliente y perder futuras ventas. Al tener poco espacio y variedad de productos, el \textit{stock} mínimo de cada ítem compite por lugar.

  \item[\textbf{Entrevistador:}] ¿Con qué frecuencia se equivocan (por exceso o por defecto)?

  \item[\textbf{Oscar Campione:}] Muy seguido. Si vendemos poco, queda mercadería que puede ponerse en mal estado. Si vendemos mucho, nos quedamos sin stock y tenemos que cortar ventas u ofrecer alternativas que el cliente no quiere, y a veces se va a otro local. Además, cuando algo faltó la noche anterior, al otro día le damos prioridad y a veces después no lo piden: terminamos comprando mal.

  \item[\textbf{Entrevistador:}] ¿Evaluaron algún software?

  \item[\textbf{Oscar Campione:}] Vimos \textit{Delivery 5.0} para integrar productos y precios y cargar pedidos manualmente, con arqueo; pero no gestiona stock.

  \item[\textbf{Entrevistador:}] ¿Les serviría un software que, usando sus ventas, prediga cantidades a comprar y permita dar de baja ítems cuando falten?

  \item[\textbf{Oscar Campione:}] Sí, ayudaría a tomar decisiones, ver qué productos se venden más y definir ofertas o promos. También mejoraría la toma de pedidos si el sistema deshabilita ítems cuando baja la mercadería, evitando avisos tardíos al cliente.

\end{description}
 % En el caso de tener entrevistas, se deben poner sus transcripciones
	\newpage
	\listoffigures
	\newpage
	\listoftables
\end{document}
