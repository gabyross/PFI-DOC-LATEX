\chapter{Introducción}

El sector gastronómico ha sido identificado como un área estratégica dentro de la economía de la Ciudad Autónoma de Buenos Aires, destacándose por su capacidad de transformar barrios a través de la generación masiva de empleos y una variada oferta culinaria, demostrando que esta actividad no solo es fundamental para el turismo, sino también para la vida cotidiana de los residentes del país \parencite{srsur2024gastronomico}. En línea con esta relevancia, la industria gastronómica en Argentina es una de las más dinámicas y resilientes a la situación económica. A pesar de que indicadores de la Cámara Argentina de Comercio y Servicios (CAC) \parencite{cac2024consumo} determinaron que el consumo en el país cerró el 2024 con una caída acumulada de -7,4\%, en el presente el sector gastronómico experimenta un incremento constante que apunta hacia la recuperación. Pero aun cuando tanto empresarios como operadores culinarios han utilizado estrategias financieras y alternativas creativas para tratar de superar las barreras que impone la inestabilidad económica del país, sigue siendo necesario encontrar mecanismos que eleven la eficiencia operativa y así conseguir un punto de equilibrio entre la rentabilidad y la sostenibilidad.

Por lo tanto, ante el panorama presentado los distintos establecimientos gastronómicos buscan la optimización de sus operaciones, apalancándose en soluciones tecnológicas que permitan gestionar aspectos tales como la venta y facturación de sus productos, la gestión de inventario, o análisis y reportes del funcionamiento del negocio, es decir, la tecnología pasa a tener un rol preponderante, ayudando a la toma de decisiones informadas que permitan maximizar la ganancia y reducir los costos \parencite{agrawal2023predictive}.

En el presente contexto, los avances relacionados a la tecnología predictiva, especialmente aquellos que comprenden modelos de Machine Learning, son capaces de brindar una solución innovadora. Como lo demuestra \parencite{schmidt2022mlsales}, los restaurantes tienen la capacidad de adaptarse a estas invenciones para aprovechar dichos modelos y así predecir la demanda.

A pesar de la implementación exponencial de la informática en múltiples áreas de la industria argentina, la predicción de ventas en el rubro gastronómico sigue sin ser explotada, especialmente por fuera de sistemas propietarios.

Brindarle esta posibilidad a un negocio significa disminuir enormemente la incertidumbre alrededor del cálculo de stock requerido. Esta es una tarea crítica, puesto que comprar por encima de lo necesario, en el caso de los ítems orgánicos con una fecha de vencimiento acotada, se traduce directamente en pérdidas económicas por desperdicio. Y de forma similar, comprar por debajo continúa generando un impacto financiero negativo debido a las ventas que no pueden concretarse, pero también impacta a la reputación del negocio y la fidelización del cliente \parencite{agrawal2023predictive}.

Con fundamento en consultas efectuadas a dueños y encargados de negocios, hoy en día esta es una tarea solamente realizada a partir de la experiencia personal y basada en los datos de ventas de la última semana, no considerando aspectos externos tales como la estacionalidad o el clima, cuando se encuentra demostrado que estos son determinantes a la hora de realizar predicciones en esta industria \parencite{tanizaki2019forecasting}.

Actualmente, se encuentran dos principales líderes de mercado en Argentina, siendo estos Maxirest y Restô, los cuales a pesar de que brindan soluciones integrales para la gestión del negocio, no atacan esta problemática en particular. Se enfocan únicamente en un análisis descriptivo de las ventas o el inventario, es decir, no abordan de forma específica la problemática de predicción de almacenamiento de stock y por ende tampoco de venta.  

Maxirest es un software integral de gestión gastronómica el cual incluye funcionalidades como reportes en tiempo real, control de inventario y caja, además de ofrecer integración con plataformas de delivery como PedidosYa o Rappi, y medios de pago como MercadoPago o Payway.  

Del mismo modo, Restô es un sistema escalable de gestión gastronómica que facilita la gestión de inventario, compras, cuentas corrientes y emisión de documentos electrónicos. Igualmente, incluye múltiples módulos administrativos y contables, junto con la integración de distintos modos de pago y delivery. 

Sin embargo, ninguno de ellos comprende capacidades de predicción de demanda mediante Machine Learning, ni estimación automática de stock basado en dichas predicciones, así como tampoco permiten el ajuste dinámico del modelo a través de la retroalimentación del usuario. En consecuencia, limitan la capacidad de los negocios para la toma de decisiones basadas en estimaciones futuras, así como tampoco contemplan una personalización avanzada según el contexto de cada local.  

Ante esta situación, se propone la creación de una plataforma web que permita a un negocio, mediante el uso de un modelo de Machine Learning, obtener predicciones de ventas, y del inventario requerido para lograr satisfacerla, es decir, ir un nivel más allá de los líderes del mercado, ofreciendo un análisis predictivo mediante el uso de tecnologías emergentes basadas en Machine Learning \parencite{posch2022bayesian}.

A fin de entrenar el modelo de Machine Learning, se utilizará la información de ventas del negocio, realizando integraciones con los sistemas de POS a fin de obtenerla en tiempo real, y a su vez, se permitirá que el usuario brinde feedback sobre las predicciones, siendo este usado para continuar ajustando el modelo a las particularidades del negocio \parencite{soto2024futuro}.

La solución está orientada a PYMES gastronómicas ubicadas en CABA, que busquen reducir la incertidumbre a la hora de tomar decisiones respecto a la gestión de su stock, utilizando para esto la información de ventas.

\section{Objetivo General}

Permitir la optimización de los niveles de stock en el rubro gastronómico de Argentina en el año 2025, mediante predicción de ventas utilizando un modelo de Machine Learning, en base a la información de ventas, a fin de reducir costos y maximizar ingresos.

\noindent\textbf{Objetivos Específicos:}

\begin{itemize}
    \item Entrevistar a dueños y/o encargados de locales gastronómicos a fin de comprender el proceso de gestión de inventario y qué factores afectan a las ventas.
    \item Desarrollar integraciones con sistemas de plataformas de delivery, tales como PedidosYa, a fin de obtener y unificar las ventas en tiempo real.
    \item Desarrollar un modelo de Machine Learning que permita pronosticar ventas y extrapolar el inventario mínimo necesario para satisfacer la demanda.
    \item Usar fuentes externas de datos, tales como APIs meteorológicas o de feriados, a fin de enriquecer los datos a usar para la predicción.
    \item Permitir que el usuario defina alertas que notifique cuando un ítem del inventario queda por debajo de un umbral.
    \item Permitir que el usuario brinde feedback sobre estos pronósticos, a fin de permitir que el modelo se ajuste.
    \item Desarrollar una plataforma web donde el usuario pueda cargar los productos que vende, indicando de qué elementos del inventario se encuentran compuestos.
\end{itemize}

\section{Alcance}

El proyecto tiene como objetivo el desarrollo de una herramienta integral para la predicción de ventas y gestión de inventario en restaurantes del área de CABA, mediante la integración con sistemas POS, un modelo de Machine Learning y una interfaz web orientada al usuario gastronómico. 

El sistema contempla: 

\begin{itemize}
    \item Integración con sistemas de POS para la recopilación unificada de datos de ventas. 
    \item Desarrollo y entrenamiento de un modelo predictivo de ventas utilizando Machine Learning. 
    \item Incorporación de variables externas y feedback de usuarios en el entrenamiento del modelo. 
    \item Página web para la visualización de ventas, predicciones e inventario. 
    \item Alertas sobre los niveles de inventario. 
\end{itemize}

En el primer release se incluyen los siguientes módulos: 

\begin{itemize}
    \item Módulo de Integración POS (Módulo 1): Desarrolla interfaces para unificar la información proveniente de los sistemas de POS más relevantes en CABA. 
    \item Módulo de Predicción de Ventas (Módulo 2): Entrena un modelo de Machine Learning con datos históricos de ventas y factores externos (clima, feriados, día, etc), considerando el feedback brindado por los usuarios a fin de ajustar aún más el modelo. 
    \item Módulo Web de Visualización (Módulo 3): Permite a los usuarios visualizar ventas históricas, predicciones, niveles de inventario requeridos y brindar feedback sobre las predicciones. 
    \item Módulo de Configuración de Productos (Módulo 4): Permite definir productos, asociar sus componentes de inventario y configurar los umbrales a utilizar por las alertas.
\end{itemize}

Sin embargo, es importante acotar que existen varios aspectos que no son incluidos dentro del alcance del presente proyecto, en vista de la modalidad y características del mismo. No comprende la integración con todos los sistemas POS disponibles, ya que se concentra únicamente en los utilizados dentro del sector gastronómico de CABA. Estos no sufren modificaciones o personalizaciones que alteren su funcionamiento.

Se limita al desarrollo de una plataforma web, por lo tanto, no incluye el diseño de una aplicación móvil o de otro tipo de software adicional. Asimismo, el modelo de Machine Learning se aplica solamente a productos gastronómicos. Por lo tanto, quedan excluidos otros tipos de negocios o productos que, aunque tengan características semejantes, no forman parte de este sector.
