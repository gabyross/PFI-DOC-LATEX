\chapter*{Resumen}

El presente Proyecto Final de Ingeniería propone el desarrollo de una plataforma web predictiva para la gestión de inventario en locales gastronómicos de la Ciudad Autónoma de Buenos Aires. El objetivo principal es optimizar el stock de inventario en PYMES gastronómicas durante 2025, mediante la predicción de ventas basada en un modelo de \emph{Machine Learning}, reduciendo pérdidas económicas por sobrestock o faltantes, y mejorando la eficiencia operativa.

La propuesta consiste en una plataforma web orientada a PYMES gastronómicas, que integra datos de ventas provenientes de plataformas como PedidosYa, y utiliza modelos de aprendizaje automático para estimar la demanda futura. A diferencia de soluciones comerciales existentes en el mercado argentino, el sistema permite incorporar variables externas (clima, feriados) y feedback del usuario para ajustar sus predicciones.


%% Si se quiere enviar este resumen en una p\'agina a alguien, descomentar esto.

%\begin{center}
%\textbf{\large Universidad Argentina De la Empresa}

%\textbf{Facultad de Ingenier\'{\i}a y Ciencias Exactas}

%\textbf{Departamento de Tecnolog\'{\i}a Inform\'atica}

%\end{center}

%\begin{center}
%\large Ingenier\'{\i}a en Inform\'atica
%\end{center}

%\begin{center}
%\textbf{\large Título de la PFI}
%\end{center}

%\begin{center}
%Apellido, Nombres
%Apellido, Nombres
%\end{center}

%\noindent 
%Acá va el resumen. \NicoNI{Este es un comentario de margen. Tener en cuenta que el breve margen que tenemos, anula un poco su utilidad. Son libres de usarlo de todas formas}

%Con el siguiente comando se imprime el interlineado en el documento pdf. \\
%Interlineado simple: 12pt.\\
%Interlineado esperado: 18pt.\\
%Interlineado exacto de este documento: \the\baselineskip.