\anexo{Resultados de la encuesta}

Con el objetivo de comprender los hábitos de consumo en locales gastronómicos y la percepción de los clientes respecto a la disponibilidad de los productos, se realizó una encuesta a más de 150 personas. El propósito de la misma fue poder identificar cómo la falta de stock y la planificación de inventario influyen directamente en la satisfacción y fidelización de los consumidores, un factor clave en el mundo gastronómico.

En cuanto a la frecuencia de consumo, el 69.8\% de los encuestados afirmó visitar locales gastronómicos o pedir por delivery al menos una vez por semana, lo que refleja una relación constante con este tipo de establecimientos.

Respecto a la disponibilidad de productos, el 47.8\%, es decir casi la mitad de los entrevistados señaló que en el último mes le sucedió que el plato o producto deseado no se encontraba disponible. Ahora bien, la falta de stock recurrente sí reflejó tener un impacto notable en la percepción del cliente ya que el 86,8\% indicó que reduciría sus visitas si un local no mantiene la disponibilidad de los platos que ofrece. Además, mientras que el 30,2\% expresó que esta situación le genera desconfianza y afecta negativamente su experiencia, el 62,92\% afirmó que le molesta al menos un poco cuando un plato no está disponible. Si se consideran ambos grupos, puede observarse que más del 93\% de los encuestados experimenta algún grado de molestia o descontento frente a la falta de disponibilidad, lo que evidencia el peso crítico de este factor en la satisfacción del cliente.

Por otra parte, el rol de la planificación de compras resultó clave: el 98,7\% de los participantes coincidió en que una mejor gestión del inventario puede mejorar significativamente el servicio. En la misma línea, el 99,4\% declaró que estaría más dispuesto a regresar a un local que siempre mantenga su menú disponible y con calidad constante, y el 98,7\% lo recomendaría más a terceros.

Finalmente, la encuesta mostró que la falta frecuente de platos afecta directamente la reputación de los locales (71,7\% lo considera un factor muy relevante) y entre las respuestas abiertas, los clientes destacaron como principal fuente de satisfacción la combinación de calidad, disponibilidad, buena atención y relación precio-calidad, confirmando la importancia de un sistema que permita optimizar el control de insumos.