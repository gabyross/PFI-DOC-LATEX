% ============================
% Anexo: Entrevista 1 (Ulises)
% ============================
\anexo{Transcripción de la entrevista a Ulises Litterio (La Brava Burger)}

Ambas entrevistas fueron realizadas en Mayo de 2025.
% Contexto: Entrevista estructurada sobre gestión de inventario y demanda en gastronomía.
% Fecha de referencia declarada por el entrevistado: mayo 2025 (para valores de costos).
\begin{description}[leftmargin=0cm, labelsep=0.5cm]

  \item[\textbf{Entrevistador:}] ¿Podrías describir tu negocio, qué venden, cuántos locales y cuántos empleados?

  \item[\textbf{Ulises Litterio:}] El negocio es comercializar hamburguesas, con producción desde la compra de la materia prima, elaboración y venta con \textit{deliveries} propios. Actualmente tenemos 3 locales, con un total de 25 empleados entre todos. Contamos con varios proveedores para cada materia prima. Nuestros locales funcionan 6 días a la semana, en turnos de día y noche.

  \item[\textbf{Entrevistador:}] Puntualizando en inventario: ¿cómo lo controlan? ¿Usan software o planillas? ¿Cuándo realizan compras a proveedores?

  \item[\textbf{Ulises Litterio:}] El inventario se gestiona por local. Lo controla el jefe de cocina, quien informa al encargado, y éste lo carga en una planilla de Excel. Esa planilla la revisa el encargado de compras (el dueño). Las compras a proveedores se realizan una vez por semana, generalmente los lunes; si hay faltantes durante la semana, se repone. Los pagos son a semana vencida. Aún con organización, a veces no sabemos si efectivamente hay stock o si alcanzará para los pedidos; el jefe de cocina suele detectarlo en el momento y avisa en mostrador.

  \item[\textbf{Entrevistador:}] ¿Cómo calculan las cantidades o productos a comprar?

  \item[\textbf{Ulises Litterio:}] Nos basamos en las ventas de la semana anterior y en el momento del mes (a principio de mes suele haber más demanda). También consideramos feriados o eventos (por ejemplo, si llueve o hace mucho calor las ventas tienden a subir).

  \item[\textbf{Entrevistador:}] ¿Qué tan frecuente es el error en ese cálculo (por exceso o por defecto)?

  \item[\textbf{Ulises Litterio:}] Es común que falle. A veces sobreestockeamos para no quedarnos cortos (ya nos pasó), pero después sobra. Por ejemplo, en la semana del “día de la hamburguesa” estimamos 25 cajas de papas, pedimos 30 y terminamos usando 22: 8 de sobrestock, que compensamos la semana siguiente. Además, los proveedores manejan listas de precios por cantidad, así que comprar menos no siempre significa gastar menos (cambia el precio unitario o las condiciones).

  \item[\textbf{Entrevistador:}] ¿Cuál es el costo aproximado de reponer inventario por semana?

  \item[\textbf{Ulises Litterio:}] A la fecha, alrededor de 6 millones de pesos por cada local.

  \item[\textbf{Entrevistador:}] ¿Averiguaste algún software para gestionar?

  \item[\textbf{Ulises Litterio:}] Consulté por MaxiRest (incluye integraciones con PedidosYa y Mercado Pago), pero el costo era alto para una PyME y la información seguía quedando segmentada.

  \item[\textbf{Entrevistador:}] ¿Te serviría un software que, usando sus ventas, prediga cantidades a comprar y alerte por bajo stock?

  \item[\textbf{Ulises Litterio:}] Sí, sería muy útil para reducir el error de stock y avisar a tiempo cuando falte materia prima, sin depender tanto del control manual del jefe de cocina. La predicción debería considerar ventas e “instante del mes”. También sería clave integrar múltiples plataformas (usamos PedidosYa y Rappi), porque hoy los reportes están fragmentados y hay que consolidar todo manualmente.

\end{description}


% ==============================
% Anexo: Entrevista 2 (Oscar C.)
% ==============================
\anexo{Transcripción de la entrevista a Oscar Campione (Rotisería Familiar)}
% Contexto: Entrevista sobre operación diaria, compras y stock en rotisería familiar con delivery propio.
\begin{description}[leftmargin=0cm, labelsep=0.5cm]

  \item[\textbf{Entrevistador:}] ¿Podrías describir tu negocio, qué venden, cuántos locales y cuántos empleados?

  \item[\textbf{Oscar Campione:}] Es una rotisería familiar; vendemos productos hechos en el día. Trabajamos 100\% con delivery propio y PedidosYa. Tenemos un solo local (en parte de nuestra casa). El único empleado es el repartidor; el resto somos familiares (en total, 4 personas).

  \item[\textbf{Entrevistador:}] Sobre inventario: ¿cómo lo controlan? ¿Usan software o planillas? ¿Cada cuánto compran?

  \item[\textbf{Oscar Campione:}] Operamos día a día, así que las compras son diarias. Vendemos lo que hay y al día siguiente reponemos. Elegimos proveedores por mejor precio del momento y pagamos al contado (no compramos en cantidad). El inventario lo controlamos al terminar el turno noche, manualmente (producto por producto), y lo anotamos en un cuaderno para reponer por la mañana. No usamos soporte digital para inventario.

  \item[\textbf{Entrevistador:}] ¿Cómo calculan qué cantidades o productos comprar?

  \item[\textbf{Oscar Campione:}] Miramos lo vendido el día anterior (y a veces la semana anterior). Definimos un tope y tratamos de vender hasta ese límite; al día siguiente reponemos. Si alguien pide algo y no queda, tenemos que informarlo, lo que puede enojar al cliente y perder futuras ventas. Al tener poco espacio y variedad de productos, el \textit{stock} mínimo de cada ítem compite por lugar.

  \item[\textbf{Entrevistador:}] ¿Con qué frecuencia se equivocan (por exceso o por defecto)?

  \item[\textbf{Oscar Campione:}] Muy seguido. Si vendemos poco, queda mercadería que puede ponerse en mal estado. Si vendemos mucho, nos quedamos sin stock y tenemos que cortar ventas u ofrecer alternativas que el cliente no quiere, y a veces se va a otro local. Además, cuando algo faltó la noche anterior, al otro día le damos prioridad y a veces después no lo piden: terminamos comprando mal.

  \item[\textbf{Entrevistador:}] ¿Evaluaron algún software?

  \item[\textbf{Oscar Campione:}] Vimos \textit{Delivery 5.0} para integrar productos y precios y cargar pedidos manualmente, con arqueo; pero no gestiona stock.

  \item[\textbf{Entrevistador:}] ¿Les serviría un software que, usando sus ventas, prediga cantidades a comprar y permita dar de baja ítems cuando falten?

  \item[\textbf{Oscar Campione:}] Sí, ayudaría a tomar decisiones, ver qué productos se venden más y definir ofertas o promos. También mejoraría la toma de pedidos si el sistema deshabilita ítems cuando baja la mercadería, evitando avisos tardíos al cliente.

\end{description}
