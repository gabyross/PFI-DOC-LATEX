\anexo{Transcripción de la entrevista a Alan Turing}

% En el Alan Turing Internet Scrapbook se menciona que Turing ofreció una charla en la radio de la BBC en 1951 
% titulada "Can Computers Think?", y participó en una discusión radiofónica en 1952. 
% Aunque no hay grabaciones conocidas, los guiones (scripts) originales sí están disponibles en el Turing Archive.
% De todas formas, me pareció anecdótico pedirle a un LLM cómo "piensa" que sería esta charla.
% Modelo: GPT5
% Prompt: Piensa cómo sería una entrevista a Alan Turing sobre el paper "Could a computer think?"
\begin{description}[leftmargin=0cm, labelsep=0.5cm]

    \item[\textbf{Entrevistador:}] Profesor Turing, su artículo plantea la pregunta “¿Podría pensar una computadora?”. ¿Cuál diría que es el núcleo de su argumento?

    \item[\textbf{Turing:}] No se trata tanto de si las máquinas pueden pensar, sino de qué entendemos por “pensar”. Propongo desplazar la pregunta hacia algo más verificable: ¿puede una máquina imitar exitosamente el comportamiento inteligente humano al punto de engañar a un observador? Eso es lo que llamo el juego de imitación.

    \item[\textbf{Entrevistador:}] Muchos críticos afirman que las máquinas nunca podrán “pensar” porque carecen de conciencia o emociones.

    \item[\textbf{Turing:}] La objeción de la conciencia es interesante, pero también se aplica a otros seres humanos: no puedo probar que usted sea consciente. Solo puedo observar su comportamiento. Por tanto, es más práctico juzgar la inteligencia por lo que un sistema hace, no por lo que supuestamente es.

    \item[\textbf{Entrevistador:}] En el artículo usted responde a varias objeciones, desde las religiosas hasta las matemáticas. ¿Hay alguna que considere la más difícil de refutar?

    \item[\textbf{Turing:}] Tal vez la objeción de la “infinita variedad” —la idea de que el comportamiento humano es demasiado variado para ser reproducido por reglas finitas. Sin embargo, las matemáticas nos muestran que sistemas finitos pueden generar gran complejidad. La biología también es un buen ejemplo: el código genético es finito, pero produce toda la diversidad de la vida.

    \item[\textbf{Entrevistador:}] Usted menciona que las máquinas podrían aprender. En 1950 eso sonaba muy futurista.

    \item[\textbf{Turing:}] Sí, creo que la programación explícita no es el único camino. Una máquina podría ser diseñada para modificar sus propias instrucciones y mejorar con la experiencia. Sería, en esencia, un aprendiz.

    \item[\textbf{Entrevistador:}] Si mirara hacia el futuro, ¿cómo imagina que evolucionará esta idea?

    \item[\textbf{Turing:}] Confío en que, hacia finales del siglo, las máquinas serán capaces de participar en el juego de imitación con éxito. Tal vez aún habrá debate sobre si “piensan”, pero la práctica hará irrelevante la pregunta: simplemente conviviremos con entidades que actúan inteligentemente.

\end{description}

% Esta entrevista fue real, pero acá dejo una versión acortada
\anexo{Transcripción de la entrevista a Richard Feynman}
\begin{description}[leftmargin=0cm, labelsep=0.5cm]
    \item[\textbf{Interviewer:}] If you get hold of two magnets, and you push them, you can feel this pushing between them. Turn them around the other way, and they slam together. Now, what is it, the feeling between those two magnets?

    \item[\textbf{Feynman:}] What do you mean, "What's the feeling between the two magnets?"

    \item[\textbf{Interviewer:}] There's something there, isn't there? The sensation is that there's something there when you push these two magnets together.

    \item[\textbf{Feynman:}] Listen to my question. What is the meaning when you say that there's a feeling? Of course you feel it. Now what do you want to know?

    \item[\textbf{Interviewer:}] What I want to know is what's going on between these two bits of metal?

    \item[\textbf{Feynman:}] They repel each other.

    \item[\textbf{Interviewer:}] What does that mean, or why are they doing that, or how are they doing that? I think that's a perfectly reasonable question.

    \item[\textbf{Feynman:}] Of course, it's an excellent question. But the problem, you see, when you ask why something happens, how does a person answer why something happens? For example, Aunt Minnie is in the hospital. Why? Because she went out, slipped on the ice, and broke her hip. That satisfies people. It satisfies, but it wouldn't satisfy someone who came from another planet and who knew nothing about why when you break your hip do you go to the hospital. How do you get to the hospital when the hip is broken? Well, because her husband, seeing that her hip was broken, called the hospital up and sent somebody to get her. All that is understood by people. And when you explain a why, you have to be in some framework that you allow something to be true. Otherwise, you're perpetually asking why. Why did the husband call up the hospital? Because the husband is interested in his wife's welfare. Not always, some husbands aren't interested in their wives' welfare when they're drunk, and they're angry.
    \ldots
\end{description}