\chapter{Conclusi\'on}\label{conclussions}

\section{Resumen de aportes}

Este trabajo ha abordado la problemática de la gestión ineficiente del inventario en el sector gastronómico de la Ciudad Autónoma de Buenos Aires, poniendo especial atención en las limitaciones que enfrentan los pequeños y medianos establecimientos al no disponer de herramientas accesibles que les permitan anticipar la demanda y optimizar el uso de sus recursos. A través de un proceso integral de investigación teórica, análisis del estado del arte, trabajo de campo con actores del rubro y desarrollo técnico de una solución funcional, se ha confirmado la existencia de una necesidad concreta de incorporar tecnologías predictivas en la gestión de inventarios gastronómicos, con el fin de mejorar la rentabilidad y la sostenibilidad operativa.

La propuesta desarrollada, denominada SmartStocker, se posiciona como una plataforma web innovadora que aplica técnicas de aprendizaje automático para predecir las ventas y calcular los niveles de stock óptimos requeridos en función de dichas predicciones. A diferencia de las herramientas disponibles en el mercado argentino, que se enfocan principalmente en la administración de ventas y la generación de reportes descriptivos, SmartStocker introduce un enfoque predictivo y adaptativo capaz de incorporar variables externas como el clima, los feriados y el comportamiento histórico del consumidor para ofrecer estimaciones más precisas. De esta manera, contribuye a reducir el desperdicio de alimentos, evitar faltantes de insumos y sostener la disponibilidad del menú, factores que impactan directamente en la experiencia del cliente y en la competitividad del negocio.

A lo largo del proceso se abordaron distintas etapas metodológicas: desde la investigación del contexto del sector gastronómico y la recolección de datos empíricos mediante entrevistas y encuestas, hasta el diseño, desarrollo, validación e implementación de la solución. Las entrevistas con dueños de pymes gastronómicas evidenciaron las dificultades de controlar el inventario con métodos manuales, así como los sobrecostos asociados al error humano y la falta de integración entre plataformas. Por su parte, las encuestas a consumidores confirmaron el impacto directo que tiene la falta de stock en la satisfacción y fidelización del cliente. Estos hallazgos sirvieron como base empírica para orientar el diseño del sistema, alineando la solución con las necesidades reales del mercado.

El análisis económico desarrollado en el capítulo final demostró la viabilidad financiera del proyecto, con indicadores positivos en los tres escenarios planteados (optimista, neutral y pesimista). Los resultados del valor actual neto, la tasa interna de retorno y el período de recuperación de la inversión evidencian que la implementación de SmartStocker es rentable y sostenible a mediano plazo, tanto para sus desarrolladores como para los potenciales usuarios del sistema. Asimismo, se destaca su escalabilidad técnica y comercial, ya que la solución puede adaptarse fácilmente a diferentes tipos de negocios gastronómicos y expandirse hacia otras regiones del país.

En conclusión, el desarrollo de SmartStocker logró integrar de manera efectiva la teoría, la práctica y la innovación. El trabajo permitió comprobar que la aplicación de modelos de aprendizaje automático en el sector gastronómico no solo es técnicamente factible, sino también útil y económicamente viable. En líneas generales, este proyecto deja en evidencia el potencial de las tecnologías predictivas para mejorar la gestión de recursos en entornos reales, y marca un punto de partida para futuras implementaciones más completas. SmartStocker representa una contribución concreta al campo de la ingeniería aplicada, y refleja cómo una solución desarrollada desde un enfoque académico puede transformarse en una herramienta práctica, con impacto real en la eficiencia y la sostenibilidad de los negocios gastronómicos.

\section{Trabajo futuro}

Al tratarse de un producto mínimo viable, el desarrollo actual de SmartStocker constituye la base de una solución que puede continuar evolucionando en futuras iteraciones. Existen diversas oportunidades de mejora y expansión que permitirían consolidar la plataforma y ampliar su alcance dentro del sector gastronómico.

Una primera línea de trabajo se orienta a la ampliación geográfica del sistema. Si bien el proyecto se desarrolló tomando como referencia el contexto gastronómico de la Ciudad Autónoma de Buenos Aires, su arquitectura permite escalar fácilmente hacia otros municipios del Gran Buenos Aires. La incorporación de nuevas zonas implicaría ajustar ciertos parámetros del modelo predictivo, considerando variaciones locales en el comportamiento del consumo, la estacionalidad y la disponibilidad de insumos. Este proceso de expansión territorial no solo permitiría validar la robustez del sistema, sino también aumentar su impacto y nivel de adopción.

En paralelo, se proyecta el desarrollo de una aplicación móvil complementaria que permita acceder a las principales funciones de la plataforma desde dispositivos Android o iOS. Esta versión facilitaría el control del inventario, la consulta de métricas y la recepción de alertas en tiempo real, ofreciendo mayor flexibilidad operativa a los usuarios y favoreciendo la adopción por parte de pequeños comercios que no cuentan con equipos de escritorio en su lugar de trabajo.

Otra mejora posible se relaciona con la experiencia del usuario. A medida que la plataforma sea utilizada por más establecimientos, resultará necesario optimizar los flujos de navegación, incorporar filtros personalizados y ampliar los reportes visuales para simplificar la interpretación de los datos. Estas mejoras apuntan a mantener una interfaz intuitiva y adaptada a diferentes perfiles de usuario, desde administradores hasta encargados de cocina o compras.

También se plantea la posibilidad de fortalecer las funcionalidades actuales mediante la incorporación de módulos complementarios. Entre ellos, se considera la automatización parcial de las órdenes de compra, la detección de tendencias de consumo específicas por categoría de producto, o la generación de alertas ante variaciones anómalas en la demanda. Estas funciones permitirían mejorar la eficiencia general del sistema y brindar un soporte más completo para la toma de decisiones.

Por último, una línea de avance relevante consiste en optimizar el modelo predictivo utilizado. Con una mayor cantidad de datos históricos disponibles, se podrá evaluar la incorporación de nuevas variables contextuales, como la temperatura, eventos locales o promociones estacionales, que podrían contribuir a aumentar la precisión de las predicciones. De igual manera, se prevé implementar mecanismos de validación periódica del modelo para asegurar su desempeño en escenarios cambiantes.

En síntesis, las líneas de trabajo propuestas no buscan transformar la esencia del sistema, sino fortalecer su funcionalidad, ampliar su alcance territorial y mejorar su usabilidad. La evolución de SmartStocker dependerá en gran medida del aprendizaje obtenido a partir de su implementación real, lo que permitirá ajustar y perfeccionar el producto de manera continua, consolidándolo como una herramienta confiable, accesible y adaptable para el sector gastronómico de la región.
